\documentclass{scrartcl}
\usepackage{../../styles/style}

%\darkmode
% https://jangsookim.github.io/lectures/vscode/vscode_lecture0.html

\title{
    Introduction to Commutative Algebra \\ \Large
    --- Solution of Homework 5 ---
    }
\author{20190262 Jeongwoo Park}
\date{}
\begin{document}
    \maketitle

    \section{Solutions}

    \begin{proof}[Solution of Problem 1]
        We need a lemma.

        \begin{Lemma}\label{radical of an ideal of Z}
            Let $\langle a \rangle \trianglelefteq \mathbb{Z}$ be a non-zero ideal. If $a = \prod_{i=1}^m p_i^{e_i}$ is the prime factorization of $a$ where $p_\bullet$ are distinct prime numbers and $e_\bullet \ge 1$ are integers, then the radical of $\langle a \rangle$ is $\left \langle \prod_{i=1}^m p_i \right \rangle$.
        \end{Lemma}

        \begin{proof}
            Simple calculation shows that
            $$ \sqrt{\langle a \rangle} = \sqrt{\prod_i \langle p_i \rangle^{e_i}} = \bigcap_i \sqrt{\langle p_i \rangle^{e_i}} = \bigcap_i \langle p_i \rangle $$
            Since $p_\bullet$ are distinct, they are coprime pairwisely. Hence, the intersection becomes the produce, so we can conclude that
            $$ \bigcap_i \langle p_i \rangle = \prod_i \langle p_i \rangle = \left \langle \prod_i p_i \right \rangle $$
            This completes a proof of the lemma.
        \end{proof}

        Let $Q \trianglelefteq \mathbb{Z}$ be a non-zero primary ideal of $\mathbb{Z}$, and let $a$ be a generator of the ideal. If we write the prime factorization of $a$ as $\prod_{i=1}^m p_i^{e_i}$ where $p_\bullet$ are distinct prime numbers and $e_\bullet \ge 1$ are integers, then the radical $\langle \prod_i p_i \rangle$ must be prime because $Q$ is primary. Thus, $m=1$ and $Q = \langle p_1^{e_1} \rangle$ and this completes the proof.
    \end{proof}

    \begin{proof}[Solution of Problem 2]
        It is enough to show that the inverse image of a closed set by the map $\phi^*$ is closed, or equivalently, $(\phi^*)^{-1}(V(I))$ is closed for all ideal $I \trianglelefteq R$. I'll show that $(\phi^*)^{-1}(V(I)) = V(S \cdot \phi(I))$

        Let's consider the direct inclusion, i.e., $\subseteq$. Let $P \in (\phi^*)^{-1}(V(I))$, or equivalently, $\phi^*(P) = \phi^{-1}(P) \supseteq I$. This implies that $P \supseteq \phi(\phi^{-1}(P)) \supseteq \phi(I)$. By considering ideals generated by each of them, we have a relation $P \supseteq S \cdot \phi(I)$. Thus, $P \in V(S \cdot \phi(I))$.

        Conversely, let's assume that $P \in V(S \cdot \phi(I))$, or equivalently, $P \supseteq S \cdot \phi(I) \supseteq \phi(I)$. By taking the inverse image, we have a relation $\phi^*(P) = \phi^{-1}(P) \supseteq \phi^{-1}(\phi(I)) \supseteq I$, hence $\phi^*(P) \in V(I)$ holds. This implies that $P \in (\phi^*)^{-1}(V(I))$, and this completes the proof.
    \end{proof}

    \begin{proof}[Solution of Problem 3]
        First, let's show that the first condition implies the second one. We need a lemma.

        \begin{Lemma}\label{spectrum lemma}
            Let $\phi : A \to B$ be a ring map. For a prime ideal $\mathfrak{p} \trianglelefteq A$, it is in the image of the induced map $\phi^* : \Spec B \to \Spec A$ if and only if $\phi^{-1}(B \cdot \phi(\mathfrak{p})) = \mathfrak{p}$.
        \end{Lemma}

        \begin{proof}
            For the direct implication, let's assume that $\mathfrak{p} = \phi(\mathfrak{q})$ where $\mathfrak{q} \trianglelefteq B$ is a prime. Since $\phi^{-1}(B \cdot \phi(\mathfrak{p})) \supseteq \mathfrak{p}$, it is enough to show the reversed inclusion. Let $a \in \phi^{-1}(B \cdot \phi(\mathfrak{p}))$, then there are elements $b_i \in B$ and $a_i \in \mathfrak{p}$ such that $\phi(a) = \sum_i b_i \phi(a_i)$. Since $a_i \in \mathfrak{p} = \phi^{-1}(\mathfrak{q})$, we can deduce that $\phi(a) \in \mathfrak{q}$, i.e., $a \in \phi^{-1}(\mathfrak{q})$. Hence, the equality $\phi^{-1}(B \cdot \phi(\mathfrak{p})) = \mathfrak{p}$ holds.

            Conversely, let's assume that $\phi^{-1}(B \cdot \phi(\mathfrak{p})) = \mathfrak{p}$. Let's consider a localization at $\mathfrak{p}$, then we have a commutative diagram
            $$
            \begin{tikzcd}
                A \arrow[r, "\phi"] \arrow[d] & B \arrow[d]\\
                A_\mathfrak{p} \arrow[r, "\phi_\mathfrak{p}"] & B_\mathfrak{p}
            \end{tikzcd}
            $$
            where the vertical maps are natural and $\phi_\mathfrak{p}$ is the induced map by the universal property. It is not too hard to show that the module $B_\mathfrak{p}$ can be identified with $S^{-1}B$ under the identification $\frac{x}{s} \leftrightsquigarrow \frac{x}{\phi(s)}$, where $S = \phi(\mathfrak{p}^c)$, in particular, we can consider $B_\mathfrak{p}$ as a localized ring. Moreover, $\phi_\mathfrak{p}$ forms a ring map because this satisfies that
            $$\phi_\mathfrak{p} (1) = 1$$
            \begin{align*}
                \phi_\mathfrak{p} \left( \frac{a_1}{s_1}+\frac{a_2}{s_2} \right) &= \phi_\mathfrak{p} \left( \frac{s_2 a_1 + s_1 a_2}{s_1 s_2} \right)\\
                &= \frac{\phi(s_2 a_1 + s_1 a_2)}{s_1 s_2}\\ 
                &= \frac{s_2 \phi(a_1) + s_1 \phi(a_2)}{s_1 s_2}\\ 
                &= \frac{\phi(a_1)}{s_1} + \frac{\phi(a_2)}{s_2}\\ 
                &= \phi_\mathfrak{p} \left( \frac{a_1}{s_1} \right) + \phi_\mathfrak{p} \left( \frac{a_2}{s_2} \right)
            \end{align*}
            and
            \begin{align*} 
                \phi_\mathfrak{p} \left( \frac{a_1}{s_1} \cdot \frac{a_2}{s_2} \right) &= \frac{\phi(a_1 a_2)}{s_1 s_2}\\ 
                &= \phi_\mathfrak{p} \left( \frac{a_1}{s_1} \right) \cdot \phi_\mathfrak{p} \left( \frac{a_2}{s_2} \right)
            \end{align*}
            Because image, inverse image, sum commute with the localization, we have the identity
            $$ (\phi_\mathfrak{p})^{-1}(B_\mathfrak{p} \cdot \phi_\mathfrak{p}(\mathfrak{p}_\mathfrak{p})) = \left(\phi^{-1}(B \cdot \phi(\mathfrak{p}))\right)_\mathfrak{p} = \mathfrak{p}_\mathfrak{p} $$
            Hence, there must be a maximal ideal $\mathfrak{m} \trianglelefteq B_\mathfrak{p}$ containing $B_\mathfrak{p} \cdot \phi_\mathfrak{p}(\mathfrak{p}_\mathfrak{p})$, and the prime ideal $\phi_\mathfrak{p}^{-1}(\mathfrak{m}) \supseteq (\phi_\mathfrak{p})^{-1}(B_\mathfrak{p} \cdot \phi_\mathfrak{p}(\mathfrak{p}_\mathfrak{p})) = \mathfrak{p}_\mathfrak{p}$ must be the maximal ideal $\mathfrak{p}_\mathfrak{p}$. By the correspondence theorem, there must be a prime ideal of $B$ whose inverse image is $\mathfrak{p}$, and this completes the proof.
        \end{proof}

        The implication follows immediately from the lemma above.

        Let's consider the implication from (2) to (3). By the assumption, there is a prime ideal $\mathfrak{p} \trianglelefteq B$ such that $\phi^{-1}(\mathfrak{p}) = \mathfrak{m}$ holds. Thus, we have
        $$ \phi(\mathfrak{m}) \cdot B = \phi(\phi^{-1}(\mathfrak{p})) \cdot B \subseteq \mathfrak{p} \cdot B = \mathfrak{p} \subsetneq B $$
        and the is what we want to show.

        Now, let's show that the third one is a sufficient condition of the fourth one. Let's pick a non-zero element $x \in M$, then there is a proper ideal $I \trianglelefteq A$ such that $A \cdot x \cong A/I$ as $A$-modules since $A \cdot x$ is a non-zero simple module. Let $\mathfrak{m} \trianglelefteq A$ be a maximal ideal containing $I$, then we have a surjective map $B \otimes_A (A/I) \twoheadrightarrow B \otimes_A (A/\mathfrak{m})$ which is obtained by tensoring on the surjective map $A/I \twoheadrightarrow A/\mathfrak{m}$. Because $B \otimes_A (A/\mathfrak{m}) \cong B/(\mathfrak{m} \cdot B)$ is non-zero, hence $B \otimes_A (A/I) \cong B \otimes_A (A \cdot x)$ is non-zero. Because $B$ is flat, an injection $A \cdot x \xhookrightarrow{} M$ induces a monomorphism $B \otimes_A (A \cdot x) \xhookrightarrow{} B \otimes_A M$, thus the module $B \otimes_A M$ is non-zero. Hence, the third condition implies the fourth one.

        Let's prove that the fourth property deduces the fifth one. Since $B$ is flat and the diagram
        $$
        \begin{tikzcd}
            M \arrow[r] \arrow[d, "\cong", "x \mapsto 1 \otimes x"'] & B \otimes_A M\\ 
            A \otimes_A M \arrow[ru]
        \end{tikzcd}
        $$
        commutes, it is enough to show that the map $\phi : A \to B$ is injective. If $B \otimes_A \ker \phi = 0$, then so is $\ker \phi$ by the assumption, and this will show the injectivity of the map $\phi$. Because the tensoring by $B$ commute with kernel, we have to show that the map $B \otimes_A A \to B \otimes_A B, b \otimes a \mapsto b \otimes \phi(a)$ is injective. We have a commutative diagram
        $$
        \begin{tikzcd}
            B \otimes_A A \arrow[r] \arrow[d, "\cong", "b \otimes a \mapsto ab"'] & B \otimes_A B\\ 
            B \arrow[ur, "b \mapsto b \otimes 1"']
        \end{tikzcd}
        $$
        so it is enough to show that the map $B \to B \otimes_A B, b \mapsto b \otimes 1$ is injective. However, there is a map $B \otimes_A B \to B, b_1 \otimes b_2 \mapsto b_1 b_2$ by the universal property of tensor product because a product is bilinear. The composition
        $$ B \overset{b \mapsto b \otimes 1}{\longrightarrow} B \otimes_A B \overset{b_1 \otimes b_2 \mapsto b_1 b_2}{\longrightarrow} B $$
        is the identity, in particular, the map at the left is injective. This is what we want to show.

        Last, I'll show that the last property implies the first one. Note that the kernel of the composition $A \overset{\phi}{\to} B \to B/(\phi(I) \cdot B)$ is $\phi^{-1}(\phi(I) \cdot B)$, which contains $I$. So, the composition factor through the natural projection $A \to A/I$, i.e., there is a map $\bar \phi A/I \to B/(\phi(I) \cdot B)$ making a commutative diagram
        $$
        \begin{tikzcd}
            A \arrow[r, "\phi"] \arrow[d, "\pi_I"'] & B \arrow[d, "\pi_{\phi(I) \cdot B}"]\\
            A/I \arrow[r, "\bar \phi"'] & B/(\phi(I) \cdot B)
        \end{tikzcd}
        $$
        where the vertical maps are natural projections. Also, we have a commutative diagram
        $$
        \begin{tikzcd}
            A/I \arrow[r, "\bar \phi"] \arrow[d] & B/(\phi(I) \cdot B) \arrow[dl, "\cong"] \\
            B \otimes_{A} (A/I)
        \end{tikzcd}
        $$
        where the isomorphism is defined as $b+\phi(I) \cdot B \mapsto b \otimes (1+I)$, and the vertical is the map $x \mapsto 1 \otimes x$. By the assumption, the map $A/I \to B \otimes_{A} (A/I)$ is injective, so is $\bar \phi$. Hence, the inverse image of $0$ by $\bar \phi \circ \pi_I = \pi_{\phi(I) \cdot B} \circ \phi$, which coincide with $\phi^{-1}(\phi(I) \cdot B)$, can be calculated as
        $$ \pi_I^{-1} \left(\bar \phi^{-1} (0) \right) = \pi_I ^{-1}(0) = I $$
        Hence, $\phi ^{-1}(\phi(I) \cdot B) = I$ holds, and this completes the proof.
    \end{proof}

    \begin{proof}[Solution of Problem 4]
        Direct implication is due to the flatness of the module $B$, so the remainder is to show the converse one. Let $f:N_1 \to N_2$ and $g:N_2 \to N_3$ be two morphisms in the sequence. Note that the tensoring by a flat module commute with kernel, cokernel, and image since it is an exact functor. Assume that the sequence
        $$ N_1 \otimes_{A} B \longrightarrow N_2 \otimes_{A} B \longrightarrow N_3 \otimes_{A} B $$
        is exact. Since we have an isomorphism
        $$ 0 = \frac{\ker(g \otimes \id) + \im(f \otimes \id)}{\im(f \otimes \id)} \cong \frac{\ker g + \im f}{\im f} \otimes_{A} B $$
        Since $B$ is faithfully flat, we can conclude that the module $\frac{\ker g + \im f}{\im f}$ is zero. Therefore, $ \ker g \subseteq \im f$ holds. Also, we have an isomorphism
        $$ \frac{\im f}{\ker g} \otimes_{A} B \cong \frac{\im (f \otimes \id)}{\ker (g \otimes \id)} = 0 $$
        so $\im f = \ker g$ holds. Therefore, the original sequence is exact, and this completes the proof.
    \end{proof}

    \begin{proof}[Solution of Problem 5]
        We know that
        $$ \frac{\mathbb{Z}[x]}{\langle 2,x \rangle} \cong \frac{\mathbb{Z}[x]/\langle x \rangle}{\langle 2,x \rangle/\langle x \rangle} \cong \mathbb{Z}/2 \mathbb{Z} $$
        is a field, in particular, the ideal $\langle 2, x \rangle$ is maximal. Also, $\sqrt{Q} \ni 2, x$ implies that $P \subseteq \sqrt{Q} \subseteq \sqrt{P} = P$ holds, so the equality $\sqrt{Q} = P$ holds. From this equanity, and since $P$ is maximal, the ideal $Q$ must be $P$-primary. However, $Q$ is not a power of $P$ because
        \begin{itemize}
            \item The relation $Q \subsetneq P$ holds since $2 \in P \setminus Q$.
            \item The relation $P^2 \subsetneq Q$ holds since $P^2 = \left\langle 4, 2x, x^2 \right\rangle \subseteq Q$ and $x \in Q \setminus P^2$ hold.
        \end{itemize}
        i.e., we have a chain
        $$ P^0 \supseteq P^1 \supsetneq Q \supsetneq P^2 \supseteq P^3 \supseteq \cdots $$
        This completes the proof.
    \end{proof}
\end{document}