\documentclass{scrartcl}
\usepackage{../../style}

%\darkmode
% https://jangsookim.github.io/lectures/vscode/vscode_lecture0.html

\title{
    Introduction to Commutative Algebra \\ \Large
    --- Solution of Homework 1 ---
    }
\author{20190262 Jeongwoo Park}
\date{}

\begin{document}
    \maketitle

    \section{Terminologies}

    \begin{itemize}[labelsep=1.5em, leftmargin=8em]
        \item[$\trianglelefteq$] For a ring $R$, a relation $I \trianglelefteq R$ means that $I$ is an ideal of $R$.
        \item[\emph{rng}] A pair $(R,+,\times, 0)$ is called a \emph{rng} if $(R,+,0)$ is an Abelian group, $(R,\times)$ is a semigroup, and the usual distributive law holds. In this paper, we also suppose the commutativity of the multiplication.
        \item[\emph{rng morphism}] A group homomorphism $f:R \to S$ between rngs $R$ and $S$ is called a \emph{rng homomorphism} if $f(r_1 r_2) = f(r_1)f(r_2)$ holds for all $r_1, r_2 \in R$.
    \end{itemize}

    \section{Solutions}
    
    \begin{proof}[Solution of Problem 1]
        I need a lemma.

        \begin{Lemma}\label{ideal of a product ring}
             Let $R, S$ be rings, and $I$ be an ideal of $R \times S$. Then, there are ideals $I_R \trianglelefteq R$ and $I_S \trianglelefteq S$ such that $I = I_R \times I_S$.
        \end{Lemma}

        \begin{proof}[Proof of the Lemma~\ref{ideal of a product ring}]
            Let's consider two rng homomorphisms $\iota_R:R \to R \times S, r \mapsto (r,0)$ and $\iota_S:S \to R \times S, s \mapsto (0,s)$. They are well-defined because for the case of $\iota_R$, the equalities

            \begin{align*}
                \iota_R(r_1+r_2) &= (r_1+r_2,0) = (r_1,0)+(r_2,0) = \iota_R(r_1)+\iota_R(r_2)\\
                \iota(r_1 r_2) &= (r_1 r_2, 0) = (r_1, 0)(r_2, 0) = \iota_R(r_1) \iota_R(r_2)
            \end{align*}

            hold, and similar equalities hold for $\iota_S$. Hence, $I_R := \iota_R^{-1}(I)$ and $I_S := \iota_S^{-1}(I)$ are ideals because an inverse image of an ideal by a rng homomorphism is again an ideal of the domain.\footnote{We know the similar proposition on ring homomorphisms, but we don't use the property that the map send $1$ to $1$. So, the proof for the ring homomorphism works for rng homomorphisms.} The remainder is to show that $I = I_R \times I_S$.

            First, let's show that the LHS is contained in the RHS. For any $(r,s) \in I$, we know that $r \in I_R$ and $s \in I_S$ because $\iota_R(r) = (r,0) = (1,0)(r,s) \in I$ and similar fact holds for $s$. Thus, $(r,s) \in I_R \times I_S$.

            Second, let's show that the LHS contains the RHS. Let $(r,s) \in I_R \times I_S$, then by the definition, $(r,0), (0,s) \in I$, hence $(r,s) = (r,0) + (0,s) \in I$. This completes the proof.
        \end{proof}

        Now, let $I$ be an ideal and $I = I_R \times I_S$ where $I_R \trianglelefteq R$ and $I_S \trianglelefteq S$. We know that $I$ is a prime ideal if and only if the quotient ring
        $$ \quo{R \times S}{I} = \quo{R \times S}{I_R \times I_S} \cong \quo{R}{I_R} \times \quo{S}{I_S} $$
        is a domain. Now, let's see a lemma.

        \begin{Lemma}\label{product and domain}
             Let $R, S$ be rings. Then, $R \times S$ is a domain if and only if one of $R$ and $S$ is a domain, and the another one is trivial.
        \end{Lemma}

        \begin{proof}[Proof of Lemma~\ref{product and domain}]
            The reversed implication is clear since $A \times 0 \cong 0 \times A \cong A$ holds for any ring $A$. So, let's assume that the product ring $R \times S$ is a domain. If both $R$ and $S$ are not trivial, then $(1,0)(0,1) = (0,0)$ but $(1,0),(0,1) \neq (0,0)$, a contradiction. Hence, one of $R$ and $S$ is trivial. Since $R \times S \cong S \times R$, it doesn't matter if we assume that $S=0$. Now, we have $R \times S \cong R$ is a domain, and this completes the proof.
        \end{proof}

        Therefore, $I$ is prime if and only if one of $I_R$ and $I_S$ is prime, and the another one is the whole ring. In the other words, a prime ideal of $R \times S$ is of the form $P \times S$ or $R \times Q$ where $P \trianglelefteq R$ and $Q \trianglelefteq S$ are prime.
    \end{proof}

    \begin{proof}[Solution of Problem 2]
        \begin{enumerate}
            \item Let's consider $(R,P,S) = (\mathbb{Z}, 2\mathbb{Z}, \mathbb{Q})$. The only prime ideal of $\mathbb{Q}$ is $0$ because $\mathbb{Q}$ is a field, but $2 \mathbb{Z} \neq 0 = \mathbb{Z} \cap 0$. Hence, there is no prime ideal $Q \trianglelefteq S$ such that $P = R \cap Q$.
            
            \item Let $(R,S) = \left(\mathbb{Z}, \mathbb{Z}[i]\right)$. First, let's prove the first property. Before to do that, let's make a lemma.
            
            \begin{Lemma}\label{dimension of a PID}
                Let $R$ be a PID that is not a field. Then, $\dim R = 1$.
            \end{Lemma}

            \begin{proof}[Proof of Lemma~\ref{dimension of a PID}]
                Since $R$ is a domain, the zero ideal is a prime ideal.\footnote{It is a part of the \textbf{Problem 3}.} Also, the zero ideal is not maximal since $R$ is not a field, i.e., $\dim R \ge 1$. To show that $\dim R \le 1$, it is enough to show that there are no prime elements $r_1, r_2 \in R$ such that $r_1R \subsetneq r_2R$ because every non-zero prime ideal of a PID is generated by a non-zero element. If there are such elements, then $r_1 = r_2 q$ for some $q \in R$. In particular, we have a relation $r_1 \mid r_2 q$ which implies that $r_1 \mid q$ because $r_1 \nmid r_2$. Now, we can find $q' \in R$ such that $q = r_1 q'$. However, this gives a result $r_1 = r_2 q = r_1 r_2 q'$, or in consequence, $r_2 q' = 1$ because $R$ is a domain. However, this contradicts to the assumption that $r_2$ is a prime element. This completes the proof.
            \end{proof}

            The property holds for the zero ideal because for every domain, the zero ideal is prime, and $0 = R \cap 0$. Now, let $0 \neq P = p \mathbb{Z} \trianglelefteq R$ be a prime ideal. Then, $p \mathbb{Z}[i]$ is a proper ideal since $\mathbb{Z}[i]^\times = \{\pm 1, \pm i\}$\footnote{If $a+bi \in \mathbb{Z}[i]$ is a unit, then its complex conjugate is also a unit with the inverse $\overline{(a+bi)^{-1}}$. Hence, $(a+bi)(a-bi) = a^2+b^2$ must be a unit and its inverse muse be an element of $\mathbb{Z}[i] \cap \mathbb{R} = \mathbb{Z}$, hence $a^2+b^2 = \pm 1$. This implies that $a+bi \in \{\pm 1, \pm i\}$, and they all are units.}. Hence, there is a maximal ideal $Q \trianglelefteq \mathbb{Z}[i]$ containing $p \mathbb{Z}[i]$. If $\mathbb{Z} \cap Q$ is a prime ideal of $\mathbb{Z}$ containing $P$, then $P=\mathbb{Z} \cap Q$ holds since we have a chain of prime ideals $0 \subsetneq P \subseteq \mathbb{Z} \cap Q$ and $\dim \mathbb{Z} = 1$ by the lemma above because $\mathbb{Z}$ is a PID.

            First, let's show that $\mathbb{Z} \cap Q$ is a proper ideal of $\mathbb{Z}$. Of course, $\mathbb{Z} \cap Q$ is an additive subgroup of $\mathbb{Z}$. Also, for any element $s \in \mathbb{Z} \cap Q$ and $m \in \mathbb{Z}$, clearly $ms \in \mathbb{Z} \cap Q$ because $Q$ is an ideal of $\mathbb{Z}[i]$, and $m \in \mathbb{Z}[i]$. Also, clearly $\mathbb{Z} \cap Q$ does not contains $1$, so it is a proper ideal.

            Second, we have to show that it is a prime ideal. However, it is a proper ideal, and if $mn \in \mathbb{Z} \cap Q$, then either $m \in Q$ or $n \in Q$ because $Q$ is maximal so a prime ideal. Hence, $m$ or $n$ is an element of $\mathbb{Z} \cap Q$, i.e., $\mathbb{Z} \cap Q$ is a prime ideal. Hence, our $R$ and $S$ satisfies the first property.

            Now, let's consider the second property. Note that $2 \mathbb{Z}$ is a prime ideal and $5 \mathbb{Z}[i] = (1+2i) \mathbb{Z}[i] \cdot (1-2i) \mathbb{Z}[i]$. Of course, $(1+2i) \mathbb{Z}[i] \neq (1-2i) \mathbb{Z}[i]$ because $\langle 1 \pm 2i \rangle$ contains $(1-2i)(1-2i) - 2 (1+2i)+(1-2i) = 1$, but $(1 \pm 2i) \mathbb{Z}[i]$ is not unit because each generator is not a unit element. As Abelian groups, we have
            $$ \quo{\mathbb{Z}[i]}{(1+2i) \mathbb{Z}[i]} \cong \quo{\mathbb{Z} \oplus \mathbb{Z}}{(1,2) \mathbb{Z} \oplus (-2,1) \mathbb{Z}}$$
            If we consider a group homomorphism $\varphi: \mathbb{Z} \oplus \mathbb{Z} \to \mathbb{Z}/5\mathbb{Z}, (m,n) \mapsto 2m-n+5\mathbb{Z}$, then it is not too hard to show that it is well-defined surjective map with kernel $(1,2)\mathbb{Z} \oplus (-2,1)\mathbb{Z}$. By the first isomorphism theorem for groups, we can conclude that
            $$ \quo{\mathbb{Z}[i]}{(1+2i) \mathbb{Z}[i]} \cong \quo{\mathbb{Z}}{5\mathbb{Z}} $$
            as Abelian groups. However, every cyclic group has the unique ring structure due to the distributive law, so the isomorphic relation above can be viewed as a ring isomorphic relation. Therefore, $(1+2i) \mathbb{Z}[i]$ is a maximal ideal because $\mathbb{Z}/5 \mathbb{Z}$ is a field, and similarly, so is $(1-2i) \mathbb{Z}[i]$. By the argument above, we can know that $5 \mathbb{Z} = \mathbb{Z} \cap (1 \pm 2i) \mathbb{Z}[i]$. Now, it is enough to show that there are only finitely many prime ideal of $\mathbb{Z}[i]$ containing $5 \mathbb{Z}$. Let's assume that $Q \trianglelefteq \mathbb{Z}[i]$ be a prime ideal containing $5 \mathbb{Z}$, then it also contains $5 \mathbb{Z}[i] = (1+2i) \mathbb{Z}[i] \cdot (1-2i) \mathbb{Z}[i]$. Here, we need a lemma.

            \begin{Lemma}\label{another definition of the prime ideal}
                 Let $R$ be a ring and $P$ be an ideal of $R$. Then, $P$ is prime if and only if for any ideals $I_1, I_2 \trianglelefteq R$, the relation $I_1 I_2 \subseteq P$ implies that one of $I_1$ and $I_2$ is contained in $P$.
            \end{Lemma}

            \begin{proof}[Proof of Lemma~\ref{another definition of the prime ideal}]
                First, let's consider the direct implication. If $P$ is prime and $I_1, I_2 \not \subseteq P$, then we can pick elements $r_i \in I_i \setminus P$ where $i \in \{1,2\}$. Then, $r_1r_2 \notin P$ since $P$ is prime, and this means that $I_1 I_2 \not \subseteq P$. This shows the direct implication.

                For the reversed implication, let's assume that $I_1 I_2 \subseteq P$ implies that one of $I_1$ and $I_2$ is contained in $P$. Let $r_1 r_2 \in P$ and pick $I_i = r_i R$ where $i \in \{1,2\}$. Then, $I_1 I_2 = r_1r_2 R \subseteq P$ holds, so either $I_1 \subseteq P$ or $I_2 \subseteq P$ holds, i.e., one of $r_1$ and $r_2$ is an element of $P$. This completes the proof.
            \end{proof}

            By the lemma above, $Q$ must contain one of $(1 \pm 2i) \mathbb{Z}[i]$. By the maximality of ideals, we can conclude $Q$ coincides with one of $(1 \pm 2i) \mathbb{Z}[i]$. Hence, there are only finitely many prime ideal of $\mathbb{Z}[i]$ containing $5 \mathbb{Z}$. This completes the solution.

            \item Let $k$ be an infinite field, and pick $(R,S) = (k, k[x])$. Clearly, both are domains and the only prime ideal of $R$ is $0$. Let's define ideals $I_a := (x-a) k[x]$ for all $a \in k$, then it is clear that they are all distinct. Also, we define ring homomorphisms $\varphi_a:k[x] \to k, f(x) \mapsto f(a)$. They are well-defined because
            
            \begin{align*}
                \varphi_a(1) &= 1\\
                \varphi_a \left( \sum_i a_i x^i + \sum_j b_j x^j \right) &= \varphi \left( \sum_i (a_i+b_i) x^i \right) \\
                &= \sum_i (a_i+b_i) a^i \\
                &= \sum_i a_i a^i + \sum_j b_j a^j = \varphi_a \left( \sum_i a_i x^i \right) + \varphi_a \left( \sum_j b_j x^j \right)
            \end{align*}

            and

            \begin{align*}
                \varphi_a \left( \left( \sum_i a_i x^i \right) \cdot \left( \sum_j b_j x^j \right) \right) &= \varphi_a \left( \sum_n \left( \sum_{i+j=n} a_i b_j \right) x^n \right)\\
                &= \sum_n \left( \sum_{i+j=n} a_i b_j \right) a^n \\ 
                &= \left( \sum_i a_i a^i \right) \cdot \left( \sum_j b_j a^j  \right) = \varphi_a\left( \sum_i a_i x^i \right) \cdot \varphi_a\left( \sum_j b_j x^j \right)
            \end{align*}

            hold. Also, the kernel of $\varphi_a$ is exactly $I_a$ because $f(a) = 0$ is equivalent to $x-a \mid f(x)$. Of course, each $\varphi_a$ is surjective, so we can deduce that $k[x]/I_a \cong k$ by the first isomorphism theorem, in particular $I_a$ is maximal thus prime. However, $k$ is infinite, so there are infinitely many $I_a$'s. Also, it is clear that $k \cap I_a = 0$. Therefore, there are infinitely many prime ideals $I_a$ such that $0 = k \cap I_a$, and this completes the solution.
        \end{enumerate}
    \end{proof}

    \begin{proof}[Solution of Problem 3]
        It is enough to show that $R/0 \cong R$ because it this holds, then $0$ is prime is equivalent to that $R/0 \cong R$ is a domain. We consider a ring homomorphism $\id: R \to R$, then it is clearly well-defined since
        
        \begin{align*}
            \id(1) &= 1\\
            \id(r_1+r_2) &= r_1+r_2 = \id(r_1)+\id(r_2)\\ 
            \id(r_1 r_2) &= r_1 r_2 = \id(r_1) \id(r_2)
        \end{align*}

        hold. Also, it has the kernel $0$, hence by the first isomorphism theorem, $R/0 \cong R$. This completes the proof.
    \end{proof}

    \begin{proof}[Solution of Problem 4]
        \begin{enumerate}
            \item Let $R$ as in the assumption in the proposition. Since $R$ is a domain, $0$ is a prime ideal by the previous problem in this homework set. By the definition of the Krull dimension, $0$ must be a maximal ideal, hence $R \cong R/0$\footnote{I proved this isomorphism in the solution of \textbf{Problem 3}.} have to be a field.
            
            \item Let's consider a ring $R = \mathbb{Z}/4 \mathbb{Z}$. There are exactly three subgroups $0, 2 \mathbb{Z}/ 4 \mathbb{Z}, R$ of $R$, and they all are ideals because $R$ has the zero ideal and the unit ideal, and the unique non-zero element of $2 \mathbb{Z} / 4 \mathbb{Z}$ is $2+4 \mathbb{Z}$ and $(2+ 4 \mathbb{Z}) \cdot (m + 4 \mathbb{Z}) = 2m + 4 \mathbb{Z} \in 2 \mathbb{Z}/4 \mathbb{Z}$ holds. We know that $R$ is not a domain because $(2+\mathbb{Z})^2 = 0$ but $2+\mathbb{Z} \neq 0$. By the previous problem in this homework set, the zero ideal is not prime. Also, the unit ideal is not prime, but $2 \mathbb{Z}/ 4 \mathbb{Z}$ must be prime because it is a maximal ideal. Therefore, the maximal length of chains of prime ideals of $R$ is exactly $0$, i.e., $\dim R = 0$.
            
            \item By the \textbf{Lemma~\ref{dimension of a PID}}, $\dim \mathbb{Z} = 1$ since $\mathbb{Z}$ is a PID that is not a field.
            
            \item By the \textbf{Lemma~\ref{dimension of a PID}}, $\dim \mathbb{Z}[i] = 1$ since $\mathbb{Z}[i]$ is an ED so a PID, and it is not a field.
            
            \item By the \textbf{Lemma~\ref{dimension of a PID}}, $\dim k[x] = 1$ since $k[x]$ is an ED so a PID, but it is not a field because $x$ is a non-zero non-unit element.
        \end{enumerate}
    \end{proof}

    \begin{proof}[Solution of Problem 5]
        \begin{enumerate}
            \item Let's take $(R,S) = (\mathbb{Z}, \mathbb{Q})$ and the natural inclusion $\phi:R \to S$. Then, $\dim \mathbb{Q} = 0 < \dim \mathbb{Z} = 1$ holds.
            
            \item Take any rings $R=S$ (e.g. $\mathbb{Z}$) and the identity map $\phi = \id :R \to S$, then clearly $\dim R = \dim S$.
            
            \item Let $(R,S) = (k, k[x])$ where $k$ is a field, and $\phi: R \to S$ be the natural inclusion. Then, $\dim k[x] = 1 > \dim k = 0$ because for any field, $0$ is the unique prime ideal.
        \end{enumerate}
    \end{proof}

    \begin{proof}[Solution of Problem 6]
        It is enough to show that for a chain
        $$ P_0 \subsetneq P_1 \subsetneq \cdots \subsetneq P_d $$
        of prime ideals of $S$, there is a chain of prime ideals of $R$ with length $d$. Let's consider a chain
        $$ \phi^{-1}(P_0) \subseteq \phi^{-1}(P_1) \subseteq \cdots \subseteq \phi^{-1}(P_d) $$
        Each inverse image is also prime, so the chain is consist of prime ideals of $R$. The remainder is to show that each inclusion of the induced chain is proper. The surjectivity of the map $\phi$ guarantees this property because if $\phi^{-1}(A) = \phi^{-1}(B)$, then we have $A = \phi(\phi^{-1}(A)) = \phi(\phi^{-1}(B)) = B$. This completes the proof.
    \end{proof}

    \begin{proof}[Solution of Problem 7]
        There is an one-to-one correspondence

        \begin{align*}
            \Hom_{k\textsf{-alg}}\left(k\left[x,\tfrac{1}{x}\right], R\right) &\overset{1-1}{\longleftrightarrow} R^\times\\ 
            f &\, \overset{\varphi}{\longmapsto} f(x)\\ 
            \left[ \sum_i a_i x^i \mapsto \sum_i a_i u^i \right]& \overset{\psi}{\longmapsfrom} u
        \end{align*}
        
        Each of $\varphi$ and $\psi$ is well-defined since $\varphi(x)$ is a unit with inverse $f\left( \frac{1}{x} \right)$, and $\psi(u)$ satisfies
        
        \begin{align*}
            \psi(u)(a \in k) &= a\\ 
            \psi(u)\left( \sum_i a_i x^i + \sum_j b_j x^j \right) &= \psi(u)\left( \sum_i (a_i+b_i) x^i \right)\\ 
            &= \sum_i (a_i+b_i) u^i \\ 
            &= \sum_i a_i u^i + \sum_j b_j u^j = \psi(u)\left( \sum_i a_i x^i \right) + \psi(u)\left( \sum_j b_j x^j \right)
        \end{align*}

        and
        
        \begin{align*}
            \psi(u) \left( \left( \sum_i a_i x^i \right) \cdot \left( \sum_j b_j x^j \right) \right) &= \psi(u) \left( \sum_n \left( \sum_{i+j=n} a_i b_j \right) x^n \right)\\ 
            &= \sum_n \left( \sum_{i+j=n} a_i b_j \right) u^n\\ 
            &= \left( \sum_i a_i u^i \right) \cdot \left( \sum_j b_j u^j \right) = \psi(u) \left( \sum_i a_i x^i \right) \cdot \psi(u)\left( \sum_j b_j x^j \right)
        \end{align*}

        hold. The remainder is to show that $\varphi$ and $\psi$ are inverse to each other. However, simple calculation shows that

        \begin{align*}
            (\psi \circ \varphi)(f) &= \psi(f(x)) \\ 
            &= \left[ \sum_i a_i x^i \mapsto \sum_i a_i f(x)^i = f\left( \sum_i a_i x^i \right) \right] = f\\
            (\varphi \circ \psi)(u) &= \psi(u)(x) = u
        \end{align*}

        and this completes the solution.
    \end{proof}

\end{document}