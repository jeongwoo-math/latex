\documentclass{scrartcl}
\usepackage{../../styles/style}

%\darkmode
% https://jangsookim.github.io/lectures/vscode/vscode_lecture0.html

\title{
    Introduction to Commutative Algebra \\ \Large
    --- Solution of Homework 4 ---
    }
\author{20190262 Jeongwoo Park}
\date{}

\begin{document}
    \maketitle

    \section{Terminologies}

    \begin{itemize}[labelsep=1.5em, leftmargin=8em]
        \item[$\Tors$] Torsion of a module
        \item[$\MaxSpec$] Maximum spectrum of a ring, i.e., the set of all maximal ideals.
    \end{itemize}

    \section{Solutions}
    
    \begin{proof}[Solution of Problem 1]
        For the reversed implication, let $\frac{x}{a} \in S^{-1}M$. Then, we have $\frac{x}{a} = s \cdot \frac{x}{as} \in sM = 0$, i.e., $S^{-1}M = 0$ holds.

        For the forward direction, assume that $M$ is generated by a finite set $\{x_i\}_{1 \le i \le m}$. Since $S^{-1}M = 0$, we have $\frac{x_i}{1} = 0$ for each $i$, or equivalently, there is an element $s_i \in S$ such that $s_i x_i = 0$. If we take $s = \prod_{i=1}^m s_i \in S$, then $s x_i = 0$ for all $i$, so we have $sM = 0$ since $\{x_i\}_i$ generated the module $M$. This completes the proof.
    \end{proof}

    \begin{proof}[Solution of Problem 2]
        \begin{enumerate}
            \item We have to show that $\Tors M$ is an additive subgroup closed under the scalar multiplication.
            
            First, let's show that the torsion of a module forms a subgroup. Of course, $\Tors M$ is non-zero since it contains the zero element since the ring $A$ is non-trivial. Let $x_1,x_2 \in \Tors M$, then there are non-zero elements $a_1, a_2 \in A$ such that $a_1 x_2 = a_2 x_2 = 0$. Since $A$ is an integral domain, we have $a = a_1 a_2 \neq 0$, and the equality $a(x_1 - x_2) = a_2 (a_1 x_1) - a_1 (a_2 x_2) = 0$ holds. Hence, $x_1 - x_2 \in \Tors M$, and we can deduce that the torsion is an additive subgroup.

            Second, we have to show that it is closed under the scalar multiplication. However, for any $x \Tors M$ such that $ax = 0$ for some non-zero $a \in A$ and $b \in A$, we have $a(bx) = b(ax) = 0$, i.e., $bx \in \Tors M$. Hence, $\Tors M$ is closed under the scalar multiplication, and this completes the proof.

            \item Suppose that $x+\Tors M$ is annihilated by a non-zero element $a \in A$, i.e., $ax \in \Tors M$. Then, there is a non-zero $b \in A$ such that $bax = 0$, but this implies that $x \in \Tors M$ because $ba \neq 0$ since the ring is a domain. Thus, $x+\Tors M = 0$, and so the quotient module $M / \Tors M$ is torsion-free.
            
            \item We have to show that for any $x \in \Tors M$, the relation $f(x) \in \Tors N$ holds. By the assumption, there is a non-zero $a \in A$ such that $ax = 0$. By taking the map $f$, we have $a f(x) = f(ax) = 0$, and this proves the relation $f(x) \in \Tors N$. Hence, $f(\Tors M) \le \Tors N$, and this is what we want to show.
            
            \item Let $S = A \setminus 0$, then we have an isomorphism $\varphi : K \otimes_A M \to S^{-1}M, \frac{a}{b} \otimes x \mapsto \frac{ax}{b}$. Because there is a commutative diagram
            $$
            \begin{tikzcd}[column sep = tiny]
                M \arrow[rr, "x \mapsto 1 \otimes x"] \arrow[dr, "x \mapsto \frac{x}{1}"'] && K \otimes_A M \arrow[dl, "\varphi"]\\
                & S^{-1}M
            \end{tikzcd}
            $$
            Hence, the kernel of the map $M \to K \otimes_A M$ coincides with the kernel of the natural map $M \to S^{-1}M$. However, the equality $\frac{x}{1} = 0$ is equivalent to that there is an element $s \in S = A \setminus 0$ such that $sx = 0$, in the other words, $x \in \Tors M$. Hence, the kernel of the natural map is $\Tors M$, and this completes the proof.

            \item Note that there is an exact sequence
            $$
            \begin{tikzcd}
                0 \arrow[r] & \Tors M \arrow[r] & M \arrow[r] & K \otimes_A M \arrow[r] & 0
            \end{tikzcd}
            $$
            Since the localization functor is exact, we have an exact sequence
            $$
            \begin{tikzcd}
                0 \arrow[r] & S^{-1} (\Tors M) \arrow[r] & S^{-1}M \arrow[r] & S^{-1}(K \otimes_A M) \arrow[r] & 0
            \end{tikzcd}
            $$
            Note that there is an isomorphism $S^{-1}K \otimes_{S^{-1}A} S^{-1}M \to S^{-1}(K \otimes_A M), \frac{a}{s} \otimes \frac{x}{s'} \mapsto \frac{a \otimes x}{ss'}$, and the natural map $K \to S^{-1} K$ is an isomorphism since it is injective since $A$ is a domain, and surjective since $\frac{a}{s}$ is an image of $s^{-1}a$. Hence, we can induce an isomorphism $K \otimes_{S^{-1} A} S^{-1}M \to S^{-1}(K \otimes_A M), \frac{a}{b} \otimes \frac{x}{s} \mapsto \frac{ax}{bs}$. Also, we have a commutative diagram
            $$
            \begin{tikzcd}
                0 \arrow[r] & S^{-1} (\Tors M) \arrow[r] \arrow[d, no head, double] & S^{-1}M \arrow[r] \arrow[d, no head, double] & K \otimes_{S^{-1}A} S^{-1}M \arrow[r] \arrow[d, "\cong"] & 0 \\
                0 \arrow[r] & S^{-1} (\Tors M) \arrow[r] & S^{-1}M \arrow[r] & S^{-1}(K \otimes_A M) \arrow[r] & 0
            \end{tikzcd}
            $$
            Hence, the kernel of the map $S^{-1}M \to K \otimes_A S^{-1}M$, which coincides to $\Tors (S^{-1}M)$ by the previous problem, is exactly the submodule $S^{-1}(\Tors M)$. This completes the proof.
        \end{enumerate}
    \end{proof}

    \begin{proof}[Solution of Problem 3]
        Since the localization and torsion commutes, the first condition implies the second one. Clearly, the second statement implies the third one, so the remainder is to show that the third one implies the first one.

        Note that a module $M$ is torsion-free if and only if the map $M \to K \otimes_A M$ is injective. The diagram
        $$
        \begin{tikzcd}[column sep = tiny]
            M_\mathfrak{m} \arrow[rr, "\frac{x}{s} \mapsto \frac{1 \otimes x}{s}"] \arrow[dr, "\frac{x}{s} \mapsto 1 \otimes \frac{x}{s}"'] && (K \otimes_A M)_\mathfrak{m} \arrow[dl, "\cong"]\\
            & K \otimes_{A_\mathfrak{m}} M_\mathfrak{m}
        \end{tikzcd}
        $$
        commutes where the map $(K \otimes_A M)_\mathfrak{m} \to K \otimes_{A_\mathfrak{m}} M_\mathfrak{m}$ is defined as in the solution of \textbf{Problem 2(4)} above. Since $M_\mathfrak{m}$ is torsion-free, the map $M_\mathfrak{m} \to K \otimes_{A_\mathfrak{m}} M_\mathfrak{m}$ in the diagram is injective. Thus the map $M_\mathfrak{m} \to (K \otimes_A M)_\mathfrak{m}$ is injective for all maximal ideal $\mathfrak{m}$. Hence, we can deduce that the map $M \to K \otimes_A M$ is injective because the injectivity is a local property. This completes the proof.
    \end{proof}

    \begin{proof}[Solution of Problem 4]
        \begin{enumerate}
            \item Since $A/I$ is generated by $1+I$, the equality $(A/I)_\mathfrak{p}$ is generated by $\frac{1+I}{1}$. Hence, the equality $(A/I)_\mathfrak{p} = 0$ is equivalent to that $\frac{1+I}{1} = 0$, i.e., there is an element $a \notin \mathfrak{p}$ such that $a \in I$. In the other words, the localization is zero if and only if $\mathfrak{p} \not \supseteq I$, or equivalently, $\mathfrak{p} \notin V(I)$. Therefore, the equality $\Supp(A/I) = V(I)$ holds.
            
            \item It is enough to show that $M_\mathfrak{p} = 0$ if and only if both $M'_\mathfrak{p}$ and $M''_\mathfrak{p}$ are zero. Since the localization functor is exact, we have an exact sequence
            $$
            \begin{tikzcd}
            0 \arrow[r] & M''_\mathfrak{p} \arrow[r] & M_\mathfrak{p} \arrow[r] & M'_\mathfrak{p} \arrow[r] & 0
            \end{tikzcd}
            $$
            If $M_\mathfrak{p} = 0$, then the exact sequence becomes
            $$
            \begin{tikzcd}
            0 \arrow[r] & M''_\mathfrak{p} \arrow[r] & 0 \arrow[r] & M'_\mathfrak{p} \arrow[r] & 0
            \end{tikzcd}
            $$
            so $M'_\mathfrak{p}, M''_\mathfrak{p} = 0$. Conversely, if both $M'_\mathfrak{p}$ and $M''_\mathfrak{p}$ are zero, then the exact sequence becomes
            $$
            \begin{tikzcd}
            0 \arrow[r] & 0 \arrow[r] & M_\mathfrak{p} \arrow[r] & 0 \arrow[r] & 0
            \end{tikzcd}
            $$
            i.e., $M=0$ holds. This completes the proof.

            \item It is enough to show that $M_\mathfrak{p} = 0$ if and only if $M_{i,\mathfrak{p}} = 0$ for all $i$. Because the localization commutes with sum, we deduce that the equality $M_\mathfrak{p} = 0$ is equivalent to that $\sum_i M_{i,\mathfrak{p}} = 0$, but this is the same with $M_{i,\mathfrak{p}} = 0$ logically. Hence, $\mathfrak{p} \in \Supp M$ if and only if $\mathfrak{p} \in \bigcup_i \Supp M_i$, and this completes the proof.
            
            \item Suppose that $M = \sum_{i=1}^{m} Ax_i$ where $x_i \in M$. Note that the kernel of the epimorphism $A \twoheadrightarrow Ax_i$ is $\Ann x_i$, so $Ax_i \cong A/Ann x_i$ as $A$-modules by the first isomorphism theorem. By the results above, we have
            $$ \Supp M = \bigcup_{i} \Supp Ax_i = \bigcup_{i} \Supp \left(\frac{A}{\Ann x_i}\right) = \bigcup_{i} V(\Ann x_i) = V\left( \bigcap_i \Ann x_i \right) $$
            However, $\bigcap_i \Ann x_i = \Ann M$ because $a \in \bigcap_i \Ann x_i$ means that $ax_i = 0$ for all $i$, but this is equivalent to that $aM=0$ since $\{x_i\}_i$ generates the $A$-module $M$. Thus, $\Supp M = V(\Ann M)$, and this completes the proof.

            \item It is enough to show that $(M \otimes_A N)_\mathfrak{p} = 0$ if and only if one of $M_\mathfrak{p}$ and $N_\mathfrak{p}$ is zero. Since the localization commutes with tensor product, we have $(M \otimes_A N)_\mathfrak{p} \cong M_\mathfrak{p} \otimes_{A_\mathfrak{p}} N_\mathfrak{p}$. By the previous homework set, we know that this tensor product is zero if and only if one of $M_\mathfrak{p}$ and $N_\mathfrak{p}$ is zero since the ring $A_\mathfrak{p}$ is local. In conclusion, $(M \otimes_A N)_\mathfrak{p} = 0$ is equivalent to that one of $M_\mathfrak{p}$ and $N_\mathfrak{p}$ is zero, and this completes the proof.
        \end{enumerate}
    \end{proof}

    \begin{proof}[Solution of Problem 5]
        \begin{enumerate}
            \item Clearly, the relation $1 \in S$ holds since no prime ideal contains units. Let $a,b \in S$, then we have to show that $ab \in S$. Since $a,b \notin P_i$, we can obtain that $ab \notin P_i$ for all $i$ because $P_i$ is prime. Hence, $ab \notin S$, and this completes the proof.
            
            \item Note that there is an isomorphism of posets
            $$ \mathscr{V}' := \left\{ \text{Prime ideals of } S^{-1}A \right\} \overset{\cong}{\longleftrightarrow} \left\{ \text{Prime ideals of } A \text{ contained in } S^c \right\} =: \mathscr{V} $$
            Also, maximal ideals of $S^{-1}A$ is exactly the maximal elements of $\mathscr{V}'$, i.e., the localization of maximal elements of $\mathscr{V}$. Let $P$ be a maximal element of $\mathscr{V}$, then by the prime avoidance, the relation $P \subseteq \bigcup_i P_i$ implies that $P \subseteq P_i$ for some $i$. Hence, $P = P_i$ for some $i$, and this implies that maximal elements of $\mathscr{V}$ are exactly $\{P_i\}_i$. Therefore, there are exactly $r$ maximal ideals since $\{P_i\}_i$ is an antichain, and they are $S^{-1} P_i$'s.
        \end{enumerate}
    \end{proof}

    \begin{proof}[Solution of Problem 6]
        \begin{enumerate}
            \item Let $\mathscr{V}$ (respectively, $\mathscr{V}'$) be the poset of all prime ideals of $A$ (respectively, $S^{-1}A$). By the one-to-one correspondence, there is an injective order-preserving map $\iota:\mathscr{V}' \xhookrightarrow{} \mathscr{V}$. The Krull dimension of $A$ (respectively, $S^{-1}A$) is the supermum of the length of chains of $\mathscr{V}$ (respectively, $\mathscr{V}'$). However, if $\mathscr{C} \subseteq \mathscr{V'}$ is a chiain of $\mathscr{V'}$, then its image under $\iota$ forms a chain of $\mathscr{V}$ of the same length. Hence, $\dim S^{-1}A \le \dim A$ holds, and this completes the proof.
            
            \item We need a definition and lemmas.
            
            \begin{Definition}
                Let $S$ be a multiplicative subset of a ring $A$. Then, the set $S$ is called \emph{complete} if for any $ab \in S$, the relation $a, b \in S$ is satisfied.
            \end{Definition}

            \begin{Lemma}\label{intersection-closedness of complete multiplicative subsets}
                Let $(S_i)_{i \in I}$ be a collection of complete multiplicative subset of a ring $A$. Then, the intersection $\bigcap_i S_i$ is again a complete multiplicative subset of $A$. In particular, for any multiplicative subset $S$ of $A$, we can define a completion
                $$ \overline{S} := \bigcap_{S' \supseteq S} S' $$
                where the intersection is on the completely multiplicative subsets $S'$.
            \end{Lemma}

            \begin{proof}[Proof of Lemma~\ref{intersection-closedness of complete multiplicative subsets}]
                Clearly, $1 \in \bigcap_i S_i$. Let's consider elements $a,b \in A$. The condition $a,b \in \bigcap_i S_i$ means that for every $i$, we have $a,b \in S_i$. It is equivalent to that for every $i$, the relation $ab \in S_i$ holds because $S_i$ is complete. In the other words, this is the same with $ab \in \bigcap_i S_i$, and this completes the proof.
            \end{proof}

            \begin{Lemma}\label{properties of completion}
                 Let $S$ be a multiplicative subset of a ring $A$. Then, following properties hold.
                 \begin{enumerate}
                     \item The identity $\overline{S} = \varphi^{-1} \left(\left(S^{-1}A \right)^\times \right)$ holds, where $\varphi : A \to S^{-1}A$ is the natural ring map.
                     \item The induced map $S^{-1}A \to \overline{S}^{-1}A$ by the universal property of localization is an isomorphism.
                     \item The element $\frac{a}{s}$ is a unit of $\overline{S}A$ if and only if $a \in \overline{S}$.
                 \end{enumerate}
            \end{Lemma}

            \begin{proof}[Proof of Lemma~\ref{properties of completion}]
                Let's show the first property. Clearly, $\varphi^{-1} \left(\left(S^{-1}A \right)^\times \right)$ contains $S$ because $\frac{s}{1}$ is a unit for all $s \in S$. Suppose that $S'$ is a complete multiplicative subset containing $S$. We have to show that $S'$ contains $\varphi^{-1} \left(\left(S^{-1}A \right)^\times \right)$. Let's pick an element $a \in A$ such that $\frac{a}{1}$ is a unit in $S^{-1}A$. Then, there is an element $\frac{b}{s}$ such that $\frac{ab}{s} = 1$, or equivalently, $(ab-s)s' = 0$ for some $s' \in S$. Hence, $abs' = ss' \in S \subseteq S'$, and this induces that $a \in S'$, because of the completeness of $S'$. Hence, $\varphi^{-1} \left(\left(S^{-1}A \right)^\times \right)$ is the completion of $S$.

                Now, let's consider the second property. Note that there is a unique induced map $\tilde{\psi}:S^{-1}A \to \overline{S}^{-1}A$ making the commutative diagram
                $$
                \begin{tikzcd}[column sep = tiny]
                    A \arrow[rr, "\psi"] \arrow[dr, "\varphi"'] && \overline{S}^{-1}A \\
                    & S^{-1}A \arrow[ur, dashed, "\exists! \tilde{\psi}"']
                \end{tikzcd}
                $$
                where $\varphi, \psi$ are natural, because $\overline{S} \supseteq S$ implies that $\psi(S) \subseteq \left(\overline{S}^{-1}A \right)^\times$ holds. However, $\varphi \left(\overline{S} \right) \subseteq \left( S^{-1}A \right)^\times$ by definition, hence there is a unique induced map $\tilde{\varphi}:\overline{S}^{-1}A \to S^{-1}A$ such that
                $$
                \begin{tikzcd}[column sep = tiny]
                    A \arrow[rr, "\psi"] \arrow[dr, "\varphi"'] && \overline{S}^{-1}A \arrow[dl, dashed, "\exists! \tilde{\varphi}"] \\
                    & S^{-1}A
                \end{tikzcd}
                $$
                However, the two compositions $\tilde{\psi} \circ \tilde{\varphi}$ and $\tilde{\varphi} \circ \tilde{\psi}$ make commutative diagrams
                $$
                \begin{tikzcd}[column sep = tiny, row sep = tiny]
                    A \arrow[rr, "\psi"] \arrow[dd, "\psi"'] \arrow[dr, "\varphi"] && \overline{S}^{-1}A \arrow[dl, "\tilde{\varphi}"] & A \arrow[rr, "\psi"] \arrow[dd, "\psi"'] && \overline{S}^{-1}A \arrow[ddll, "\id"] \\ 
                    & S^{-1}A \arrow[dl, "\tilde{\psi}"] \\ 
                    \overline{S}^{-1}A &&& \overline{S}^{-1}A
                \end{tikzcd}
                $$
                and
                $$
                \begin{tikzcd}[column sep = tiny, row sep = tiny]
                    A \arrow[rr, "\varphi"] \arrow[dd, "\varphi"'] \arrow[dr, "\psi"] && S^{-1}A \arrow[dl, "\tilde{\psi}"] & A \arrow[rr, "\varphi"] \arrow[dd, "\varphi"'] && S^{-1}A \arrow[ddll, "\id"] \\ 
                    & \overline{S}^{-1}A \arrow[dl, "\tilde{\varphi}"] \\ 
                    S^{-1}A &&& S^{-1}A
                \end{tikzcd}
                $$
                By the universal property of localization, we can conclude that both $\tilde{\psi} \circ \tilde{\varphi}$ and $\tilde{\varphi} \circ \tilde{\psi}$ are the identity maps, and proves the second proposition.

                Now, let's think about the third one. The element $\frac{a}{s}$ is a unit if and only if there is another element $\frac{a'}{s'}$ such that $\frac{aa'}{ss'} = 1$, or equivalently, there is an element $s'' \in \overline{S}$ such that $(aa'-ss')s'' = 0$. Hence, the relation $aa's'' = ss's'' \in \overline{S}$ holds, and this implies that $a \in \overline{S}$, because $\overline{S}$ is complete. This is the end of the proof.
            \end{proof}

            \begin{Lemma}\label{localization of a domain}
                Let $S$ be a multiplicative subset of a domain $A$. Then, a localization $S^{-1}A$ is a field if and only if the completion $\overline{S}$ is the set $A \setminus 0$.
            \end{Lemma}

            \begin{proof}[Proof of Lemma~\ref{localization of a domain}]
                We can assume that the multiplicative set $S$ is complete by the \textbf{Lemma~\ref{properties of completion}}. Under this assumption, it is enough to show that $S = A \setminus 0$. If not, then there is a non-zero element $a \in A \setminus S$. Because $A$ is a domain, and $S$ does not contains $0$, the natural map $A \to S^{-1}A$ is injective. Hence, the element $\frac{a}{1}$ is non-zero, so it is a unit because we assumed that the localization $S^{-1}A$ is a field. It is equivalent to that there is an element $\frac{a'}{s'}$ such that $\frac{aa'}{s'} = 0$, in the other words, there is an element $s \in S$ such that $aa's = 0$. By the completeness, $a$ must be an element of $S$, and this contradicts to the assumption that $a \in A \setminus S$. Therefore, $S$ must be the set $A \setminus 0$, and this completes the proof.
            \end{proof}

            Now, let's consider a PID $\mathbb{Z}$ and its localization $\mathbb{Z}_{(p)}$. We know an inequality $\dim \mathbb{Z}_{(p)} \le \dim \mathbb{Z} = 1$ because a dimension of a PID is $1$ whenever it is not a field. Also, the localization $\mathbb{Z}_{(p)}$ is a domain but not a field by \textbf{Lemma~\ref{localization of a domain}} and the fact that $S = A \setminus \mathfrak{p}$ is always complete since if $ab \notin \mathfrak{p}$ (i.e., $ab \in S$), then $a,b \notin \mathfrak{p}$ (i.e., $a,b \in S$) holds. Hence, $\dim \mathbb{Z}_{(p)} \ge 1$ since every domain of dimension zero must be a field. Therefore, $\dim \mathbb{Z}_{(p)} = 1 = \dim \mathbb{Z}$, and this forms an example.

            \item It is enough to find a domain $A$ of dimension at least $3$, because the field of fraction of $A$ must be of the dimension $0$. Let's consider a polynomial ring $k[x_i]_{i \in \mathbb{N}}$ of infinitely many variables over a field $k$. Note that every polynomial ring is a domain, at least if it is consist of finitely many variables. However, the relation $k[x_i]_{i \in \mathbb{N}} = \bigcup_{j \in \mathbb{N}} k[x_i]_{0 \le i \le j}$ implies that $k[x_i]_{i \in \mathbb{N}}$ is also a domain. This is because if $f(x_\bullet) \cdot g(x_\bullet) = 0$, then there is a large natural number $j$ such that $f(x_\bullet), g(x_\bullet) \in k[x_i]_{0 \le i \le j}$, so one of $f(x_\bullet)$ and $g(x_\bullet)$ must be zero because the subring $k[x_i]_{0 \le i \le j}$ is a domain.
            
            I'll show that the chain
            $$ 0 \lneq \langle x_0 \rangle \lneq \langle x_0, x_1 \rangle \lneq \langle x_0, x_1, x_2 \rangle \lneq \cdots $$
            is consist of prime ideals. Let $f(x_\bullet) \cdot g(x_\bullet) \in \langle x_i \rangle_{1 \le i \le j}$, then there are polynomials $a_i(x_\bullet)$ such that
            $$ f(x_\bullet) \cdot g(x_\bullet) = \sum_{1 \le i \le j} a_i(x_\bullet) \cdot x_i $$
            We can pick large $j' > j$ such that $f(x_\bullet), g(x_\bullet) \in k[x_i]_{1 \le i \le j'}$, and by considering the evaluation $x_{j'+1}, x_{j'+2}, \cdots = 0$, we can conclude that
            $$ f(x_\bullet) \cdot g(x_\bullet) = \sum_{1 \le i \le j} a_i(x_1, \cdots, x_j, 0, \cdots) \cdot x_i \in \sum_{1 \le i \le j} k[x_i]_{1 \le i \le j'} \cdot x_i $$
            Because the ideal $\sum_{1 \le i \le j} k[x_i]_{1 \le i \le j'} \cdot x_i \trianglelefteq k[x_i]_{1 \le i \le j}$ is prime, one of $f(x_\bullet)$ and $g(x_\bullet)$ is an element of $\sum_{1 \le i \le j} k[x_i]_{1 \le i \le j'} \cdot x_i \subseteq \langle x_i \rangle_{1 \le i \le j}$, and this proves that the chain above is consist of prime ideals, in particular, $\dim k[x_i]_i = \infty$. This completes the solution.
        \end{enumerate}
    \end{proof}

    \begin{proof}[Solution of Problem 7]
        \begin{enumerate}
            \item Clearly, $1 \in S = \mathbb{Q}[x] \setminus 0$ since it is a non-zero $\mathbb{Q}$-coefficient polynomial. If we pick $a,b \in S$, then $ab \in \mathbb{Q}[x]$ is non-zero since the polynomial ring $\mathbb{Q}[x]$ is a domain. Thus, the set $S$ is multiplicative.
            
            \item By the \textbf{Lemma~\ref{localization of a domain}} and the similar argument in \textbf{Problem~6(3)}, it is enough to show that the completion of the multiplicative subset $\mathbb{Q}[x] \setminus 0$ is properly contained in $\mathbb{C}[x] \setminus 0$. Let $\alpha \in \mathbb{C}$ be a transcendental number, then the ideal $\langle x-\alpha \rangle$ is maximal, because it is a kernel of an epimorphism $k[x] \to k, f(x) \mapsto f(\alpha)$ and so $k[x]/\langle x-\alpha \rangle \cong k$. Note that the relation $f(x) \in \langle x-\alpha \rangle$ is equivalent to $f(\alpha) = 0$. From this fact, we can deduce that $S \cap \langle x-\alpha \rangle = \varnothing$, i.e., $S \subseteq \langle x-\alpha \rangle^c$. In the proof of \textbf{Problem~6(2)}, we showed that a complement of a prime ideal is a complete multiplicative subset, so $\overline{S} \subseteq \langle x-\alpha \rangle^c \subsetneq \mathbb{C}[x] \setminus 0$ holds. This completes the proof.
            
            \item Note that since $\mathbb{C}$ is algebraically closed and the polynomial ring $\mathbb{C}[x]$ is a PID, every maximal ideal of $\mathbb{C}[x]$ is of the form $\langle x-\alpha \rangle$ for some $\alpha \in \mathbb{C}$. There is a one-to-one correspondence
            $$ \left\{ \text{Transcendental numbers of } \mathbb{C} \right\} \overset{\alpha \mapsto \langle x-\alpha \rangle}{\longleftrightarrow} \left\{ \langle x-\alpha \rangle \right\}_{\alpha \text{ is transcendental.}} \subseteq \MaxSpec A $$
            because if $\langle x-\alpha \rangle = \langle x-\alpha' \rangle$, then this ideal contains $(x-\alpha)-(x-\alpha') = \alpha' - \alpha$, but this cannot be a unit, i.e., $\alpha = \alpha'$ must be hold. Also, there is a one-to-one correspondence
            $$ \left\{ \text{Maximal ideals of } S^{-1}A \right\} \longleftrightarrow \left\{ \text{Maximal elements of the poset } \mathscr{V} \right\} $$
            where $\mathscr{V}$ is the collection of all prime ideals disjoint from $S$. However, maximal elements of $\mathscr{V}$ is exactly the maximal ideals disjoint from $S$, because the only non-maximal prime ideal is $0$, but it is not a maximal element of $\mathscr{V}$ because $S^{-1}A$ is of the dimension $1$. Now, the remainder is to show that any maximal ideal is of the form $\langle x-\alpha \rangle$ for some transcendental number $\alpha \in \mathbb{C}$ if and only if the maximal ideal is disjoint from $S$.

            First, let's think about the forward direction. If $f(x) \in \langle x-\alpha \rangle \cap S$, then $f(\alpha)=0$ holds, so $\alpha$ must be algebraic. So, for any maximal ideal $\langle x-\alpha \rangle$ with transcendental $\alpha$ is disjoint from $S$.

            Second, let's assume that a maximal ideal $\langle x-\alpha \rangle$ with algebraic $\alpha$. Then, it contains the minimal polynomial $m_\alpha(x) \in S$ of $\alpha$, so $\langle x-\alpha \rangle$ is not disjoint from $S$. Hence, if $\langle x-\alpha \rangle$ is disjoint from $S$, then $\alpha$ must be transcendental. This completes thr proof.

        \end{enumerate}
    \end{proof}
    
\end{document}