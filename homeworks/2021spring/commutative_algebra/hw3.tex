\documentclass{scrartcl}
\usepackage{../../styles/style}

%\darkmode
% https://jangsookim.github.io/lectures/vscode/vscode_lecture0.html

\title{
    Introduction to Commutative Algebra \\ \Large
    --- Solution of Homework 3 ---
    }
\author{20190262 Jeongwoo Park}
\date{}

\begin{document}
    \maketitle

    \section{Solutions}
    
    \begin{proof}[Solution of Problem 1]
        It is enough to show that all decomposable element is zero, because such elements generates the tensor product. Let $\overline{m} \in \mathbb{Z}/5 \mathbb{Z}$ and $\overline{n} \in \mathbb{Z}/7 \mathbb{Z}$. Here, we can pick $m$ as an integer divisible by $7$ because $\gcd(5,7) = 1$, and so $\overline{m} \otimes \overline{n} = \overline{\frac{m}{7}} \otimes 7 \overline{n} = 0$ holds. This completes the proof.
    \end{proof}

    \begin{proof}[Solution of Problem 2]
        We need some lemmas.

        \begin{Lemma}\label{module quotient and tensoring}
             Let $M$ be an $A$-module, and let $I \trianglelefteq A$ be an ideal. Then, there is a natural ($A/I$)-module isomorphism
             $$ \frac{A}{I} \otimes_A M \cong \frac{M}{IM} $$
        \end{Lemma}

        \begin{proof}[Proof of Lemma~\ref{module quotient and tensoring}]
            By the universal property of tensor product, we have an induced $A$-linear map $\varphi : \frac{A}{I} \otimes_A M \to \frac{M}{IM}$ such that $(a+I) \otimes m \mapsto am+IM$ from the bilinear map $\frac{A}{I} \times M \to \frac{M}{IM}, \left( a+I, m \right) \mapsto am+IM$. Moreover, this map is ($A/I$)-linear. Also, we can define a ($A/I$)-linear map $\psi : \frac{M}{IM} \to \frac{A}{I} \otimes_A M, m+IM \mapsto (1+I) \otimes m$. This is well-defined as a function because if $m+IM = m'+IM$, then there are $a \in I$ and $x \in M$ such that $m-m' = ax$, but then $(1+I) \otimes m = (1+I) \otimes (m'+ax) = (1+I) \otimes m' + (a+I) \otimes x = (1+I) \otimes m'$, and the usual computation using the bilinearity of $\otimes$ shows that this function is actually a ($R/I$)-linear map. However, two ($R/I$)-linear maps $\varphi$ and $\psi$ are inverse to each other because
            $$ (\varphi \circ \psi)(m+IM) = \varphi((1+I) \otimes m) = m+IM $$
            and
            $$ (\psi \circ \varphi)((a+I) \otimes m) = \psi(am+IM) = (1+I) \otimes am = (a+I) \otimes m $$
            holds, and the decomposable elements generates the tensor product. This proves the lemma.
        \end{proof}

        \begin{Lemma}\label{change of base ring}
             Let $B$ be an $A$-algebra, and let $M,N$ be $B$-modules. Then, there is a natural $B$-module isomorphism
             $$ B \otimes_A M \otimes_A N \cong (B \otimes_A M) \otimes_B (B \otimes_A N) $$
        \end{Lemma}

        \begin{proof}[Proof of Lemma~\ref{change of base ring}]
            The map $B \times M \times N \to (B \otimes_A M) \otimes_B (B \otimes_A N), (b,m,n) \mapsto b \cdot [(1 \otimes m) \otimes (1 \otimes n)]$ is $A$-trilinear, so this induces an $A$-linear map $\varphi : B \otimes_A M \otimes_A N \to (B \otimes_A M) \otimes_B (B \otimes_A N)$ such that $b \otimes m \otimes n \mapsto b \cdot \left[(1 \otimes m) \otimes (1 \otimes n)\right]$. Clearly, it is also a $B$-linear map.

            However, it is not clear that the $B$-linear map $L : (B \otimes_A M) \times (B \otimes_A N) \to B \otimes_A M \otimes_A N, (b_1 \otimes m, b_2 \otimes n) \mapsto b_1 b_2 \otimes m \otimes n$ is well-defined. To show the well-definedness, we consider an element $b_2 \otimes n \in S \otimes_A N$ and an $A$-bilinear map $B \times M \to B \otimes_A M \otimes_A N, (b_1, m) \mapsto b_1 b_2 \otimes m \otimes n$. This induces a map $B \otimes_A M \to B \otimes_A M \otimes_A N$ such that $b_1 \otimes m \mapsto b_1 b_2 \otimes m \otimes n$. Similarly, for any element $b_1 \otimes m \in B \otimes_A M$, we can define an $A$-linear map $B \otimes_A N \to B \otimes_A M \otimes_A N, b_2 \otimes n \mapsto b_1 b_2 \otimes m \otimes n$. Therefore, if $(b_1 \otimes m, b_2 \otimes n) = (b'_1 \otimes m', b'_2 \otimes n')$, then we have
            $$ b_1 b_2 \otimes m \otimes n = b'_1 b_2 \otimes m \otimes n = b'_1 b'_2 \otimes m \otimes n $$
            by the maps we defined above. Hence, the map $L$ is well-defined $A$-linear map, and it is not too hard to show that it is actually $B$-linear. Thus, this induces a map $\psi : (B \otimes_A M) \otimes_B (B \otimes_A N) \to B \otimes_A M \otimes_A N$ such that $(b_1 \otimes m) \otimes (b_2 \otimes n) \mapsto b_1 b_2 \otimes m \otimes n$.

            Simple calculation shows that two $B$-linear maps $\varphi$ and $\psi$ are inverse to each other, and this completes the proof.
        \end{proof}

        Now, we can deduce an $k$-module isomorphism
        $$ \frac{M}{\mathfrak{m}M} \otimes_{k} \frac{N}{\mathfrak{m}M} \cong \left( \frac{A}{\mathfrak{m}} \otimes_A M \right) \otimes_k \left( \frac{A}{\mathfrak{m}} \otimes_A N \right) \cong \frac{A}{\mathfrak{m}} \otimes_A M \otimes_A N = 0 $$
        where $k$ is the residue field of the local ring $(A, \mathfrak{m})$, since $M \otimes_A N = 0$. However, this can be happen only when $M/\mathfrak{m}M = 0$ or $N/\mathfrak{m}N = 0$ holds, or equivalently, either $M=\mathfrak{m}M$ or $N = \mathfrak{m}N$ holds. This is because for any $k$-vector spaces $V$ and $W$, the dimension of the tensor product $V \otimes_k W$ is $\dim V \times \dim W$ since every vector space is free, and tensor product and direct sum satisfies the distribution law. By the Nakayama lemma, we can deduce that either $M=0$ or $N=0$ holds, and this completes the proof.
    \end{proof}

    \begin{proof}[Solution of Problem 3]
        \begin{enumerate}
            \item Let $f:P \to P'$ be a $A$-module monomorphism, then so is the map $\id \otimes (\id \otimes f) : M \otimes_A (N \otimes P) \to M \otimes_A \left(N \otimes_A P'\right)$. The diagram
            $$
            \begin{tikzcd}
                M \otimes_A (N \otimes_A P) \arrow[d, "\cong"] \arrow[r, "\id \otimes ((\id \otimes f)"] & M \otimes_A (N \otimes_A P') \arrow[d, "\cong"]\\
                (M \otimes_A N) \otimes_A P \arrow[r, "\id \otimes f"'] & (M \otimes_A N) \otimes_A P'
            \end{tikzcd}
            $$
            commutes where the vertical isomorphisms are mappings of the form $m \otimes (n \otimes p) \mapsto (m \otimes n) \otimes p$. Therefore, the map $\id \otimes f : (M \otimes_A N) \otimes_A P \to (M \otimes_A N) \otimes_A P'$ is monic, and this completes the proof.

            \item We need a lemma.

            \begin{Lemma}\label{change of base ring 0}
                Let $B$ be an $A$-module, $M$ be an $A$-module, and $N$ be a $B$-module. Then, there is a natural $B$-module isomorphism
                $$ (B \otimes_A M) \otimes_B N \cong M \otimes_A N $$
                where the $B$-module structure on $M \otimes_A N$ is given by $b(m \otimes n) := m \otimes bn$.\footnote{It is easy to show that this scalar multiplication is well-defined, by the usual argument using the universal property of tensor product.}
            \end{Lemma}
    
            \begin{proof}[Proof of Lemma~\ref{change of base ring 0}]
                Foy any $n \in N$, the $A$-bilinear map $B \times M \to M \otimes_A N, (b, m) \mapsto m \otimes bn$ induces an $A$-linear map $B \otimes_A M \to M \otimes_A N, b \otimes m \mapsto m \otimes bn$. By using this map, we can prove that the $B$-bilinear map $(B \otimes_A M) \times N \to M \otimes_A N, ((b \otimes m), n) \mapsto m \otimes bn$ is well-defined as a function, and it is not too hard to verify that it is well-defined as a $B$-bilinear map. By the universal property of tensor product, we can induce a map $\varphi : (B \otimes_A M) \otimes_B N \to M \otimes_A N, (b \otimes n) \otimes m \mapsto m \otimes bn$.

                Also, we can define an $A$-module homomorphism $\psi : M \otimes_A N \to (B \otimes_A M) \otimes_B N, m \otimes n \mapsto (1 \otimes m) \otimes n$ by using the universal property of tensor product, and this is actually $B$-linear. Simple calculation verifies that two $B$-linear maps $\varphi$ and $\psi$ are inverse to each other, and this completes the proof.
            \end{proof}

            Now, suppose that the sequence
            $$
            \begin{tikzcd}
                0 \arrow[r] & M \arrow[r, "f"] & M'
            \end{tikzcd}
            $$
            is exact in $\lmod{A}$. Since $B$ is flat as an $A$-algebra, we have an exact sequence
            $$
            \begin{tikzcd}
                0 \arrow[r] & B \otimes_A M \arrow[r, "\id \otimes f"] & B \otimes_A M'
            \end{tikzcd}
            $$
            in $\lmod{A}$. However, this can be viewed as an exact sequence in $\lmod{B}$ since the natural $B$-module structures makes the map $\id \otimes f$ to be a $B$-module homomorphism. More precisely, we have the identity
            $$ (\id \otimes f) \left(b' (b \otimes m) \right) = (\id \otimes f) \left(b'b \otimes m \right) = b'b \otimes f(m) = b' (b \otimes f(m)) = b' \cdot (\id \otimes f)(b \otimes m) $$
            holds. Hence, tensoring by $N$ gives an exact sequence
            $$
            \begin{tikzcd}
                0 \arrow[r] & (B \otimes_A M) \otimes_B N \arrow[r, "(\id \otimes f) \otimes \id"] & (B \otimes_A M') \otimes_B N
            \end{tikzcd}
            $$
            in $\lmod{B}$, because $N$ is flat as a $B$-module. However, it can be considered as a sequence in $\lmod{A}$ because $B$ is an $A$-algebra. Now, we have a commutative diagram
            $$
            \begin{tikzcd}
                0 \arrow[r] & (B \otimes_A M) \otimes_B N \arrow[r, "(\id \otimes f) \otimes \id"] \arrow[d, "\cong"] & (B \otimes_A M') \otimes_B N \arrow[d, "\cong"]\\
                0 \arrow[r] & M \otimes_A N \arrow[r, "f \otimes \id"'] & M' \otimes_A N
            \end{tikzcd}
            $$
            where the vertical isomorphisms are from the proof of the \textbf{Lemma~\ref{change of base ring 0}}. Hence, the second row is exact, and this proves the flatness of an $A$-module $N$.

            \item We can assume that $M = A^{\bigoplus I}$ for some index set $I$, and let's consider an exact sequence
            $$
            \begin{tikzcd}
                0 \arrow[r] & N \arrow[r, "f"] & N'
            \end{tikzcd}
            $$
            Tensoring by $A^{\bigoplus I}$ gives a sequence
            $$
            \begin{tikzcd}
                0 \arrow[r] & A^{\bigoplus I} \otimes_A N \arrow[r, "\id \otimes f"] & A^{\bigoplus I} \otimes_A N'
            \end{tikzcd}
            $$
            Note that the sequence
            $$
            \begin{tikzcd}
                0 \arrow[r] & N^{\bigoplus I} \arrow[r, "f^{\bigoplus I}"] & N'^{\bigoplus I}
            \end{tikzcd}
            $$
            is exact. However, the natural isomorphisms $N^{\bigoplus I} \cong A^{\bigoplus I} \otimes_A N$ and $N'^{\bigoplus I} \cong A^{\bigoplus I} \otimes_A N'$ gives a commutative diagram
            $$
            \begin{tikzcd}
                0 \arrow[r] & A^{\bigoplus I} \otimes_A N \arrow[r, "\id \otimes f"] \arrow[d, "\cong"]& A^{\bigoplus I} \otimes_A N' \arrow[d, "\cong"]\\
                0 \arrow[r] & N^{\bigoplus I} \arrow[r, "f^{\bigoplus I}"] & N'^{\bigoplus I}
            \end{tikzcd}
            $$
            This proves that every free module is flat.
        \end{enumerate}
    \end{proof}

    \begin{proof}[Solution of Problem 4]
        Note that for any $A$-module $N$, there is an epimorphism from a free module to the module $N$. This is because we can consider a surjective map $A^{\bigoplus N} \twoheadrightarrow N, (a_n)_{n \in N} \mapsto \sum_{n} a_n n$.

        Let $f_0 : F_0 \to M$ be any epimorphism such that $F_0$ is free. We can find an epimorphism $g_1 : F_1 \to \ker f_0$, and let $f_1$ is the composition $F_1 \overset{g_1}{\to} \ker f_0 \xhookrightarrow{} F_0$. Then, the kernel of $f_0$ is exactly the image of $f_1$, so we have an exact sequence
        $$
        \begin{tikzcd}
            F_1 \arrow[r, "f_1"] & F_0 \arrow[r, "f_0"] & M \arrow[r] & 0
        \end{tikzcd}
        $$
        Again, we consider an epimorphism $g_2 : F_2 \to \ker f_1$, and let $f_2$ be the composition $F_2 \overset{g_2}{\to} \ker f \xhookrightarrow{} F_1$. Then, the kernel of $f_1$ is exactly the image of $f_2$, so the sequence
        $$
        \begin{tikzcd}
            F_2 \arrow[r, "f_2"] & F_1 \arrow[r, "f_1"] & F_0 \arrow[r, "f_0"] & M \arrow[r] & 0
        \end{tikzcd}
        $$
        By recurring this process, we have a free resolution of $M$
        $$
        \begin{tikzcd}
            \cdots \arrow[r, "f_3"] & F_2 \arrow[r, "f_2"] & F_1 \arrow[r, "f_1"] & F_0 \arrow[r, "f_0"] & M \arrow[r] & 0
        \end{tikzcd}
        $$
        This completes the proof.
    \end{proof}

    \begin{proof}[Solution of Problem 5]
        We need a lemma.

        \begin{Lemma}\label{free then projective}
             Every free module is \href{https://en.wikipedia.org/wiki/Projective_module}{\uwave{projective}}.
        \end{Lemma}

        \begin{proof}[Proof of Lemma~\ref{free then projective}]
            Let $I$ be an index set, $\alpha : A^{\bigoplus I} \to N$ be a map, and $f : M \to N$ be an epimorphism.
            $$
            \begin{tikzcd}
                & A^{\bigoplus I} \arrow[d, "\alpha"] \arrow[ld, dashed, "\tilde{\alpha}"']\\
                M \arrow[r, "f"'] & N \arrow[r] & 0
            \end{tikzcd}
            $$
            We have to show that there is a map $\tilde{\alpha}$ making the diagram above commutes. Let's consider a set-map $\overline{\alpha} : I \to M$ sending $i \in I$ to any element in $f^{-1}(\alpha(\delta_{i,j})_j)$. By the universal property of free module, there is $\tilde{\alpha}$ such that $\tilde{\alpha}(\delta_{i,j})_i = \overline{\alpha}(i)$.
            $$
            \begin{tikzcd}
                I \arrow[d, "\overline{\alpha}"'] \arrow[r, "i \mapsto (\delta_{i,j})_j"] & A^{\bigoplus I} \arrow[ld, dashed, "\tilde{\alpha}"]\\
                M
            \end{tikzcd}
            $$
            We can obtain the identity
            $$ (f \circ \tilde{\alpha})(a_i)_i = \sum_i a_i \cdot (f \circ \tilde{\alpha})(\delta_{i,j})_j = \sum_i a_i \cdot f(\overline{\alpha}(i)) = \sum_i a_i \cdot \alpha(\delta_{i,j})_j = \alpha(a_i)_i $$
            This completes the proof.
        \end{proof}

        I'll show that for any \href{https://en.wikipedia.org/wiki/Resolution_(algebra)#Free,_projective,_injective,_and_flat_resolutions}{\uwave{projective resolution}} $F_{\bullet} \to M$ and any \href{https://en.wikipedia.org/wiki/Resolution_(algebra)}{\uwave{resolution}} $G_{\bullet} \to N$, any morphism $\alpha:M \to N$ has the unique lift $\tilde{\alpha}: F_{\bullet} \to G_{\bullet}$ up to chain homotopy, i.e, there are collection of maps $\tilde{\alpha}_n : F_n \to G_n$ making the commutative diagram
        $$
        \begin{tikzcd}
            \cdots \arrow[r, "f_3"] & F_2 \arrow[r, "f_2"] \arrow[d, "\tilde{\alpha}_2"] & F_1 \arrow[r, "f_1"] \arrow[d, "\tilde{\alpha}_1"] & F_0 \arrow[r, "f_0"] \arrow[d, "\tilde{\alpha}_0"] & M \arrow[r] \arrow[d, "\alpha"] & 0\\
            \cdots \arrow[r, "g_3"] & G_2 \arrow[r, "g_2"] & G_1 \arrow[r, "g_1"] & G_0 \arrow[r, "g_0"] & N \arrow[r] & 0
        \end{tikzcd}
        $$
        and if there is another lift $\widehat{\alpha}$, then there is a collection of maps $s_n : F_n \to G_{n+1}$ such that $\tilde{\alpha}_n - \widehat{\alpha}_n = f_n s_{n-1} + s_n g_{n+1}$.

        First, let's show the existence. Since $F_0$ is projective and the map $g_0$ is epic, we can lift the composition $\alpha \circ f_0$. Let's say this lift is $\tilde{\alpha}_0$.
        $$
        \begin{tikzcd}
            F_0 \arrow[r, "f_0"] \arrow[d, dashed, "\tilde{\alpha}_0"] & M \arrow[r] \arrow[d, "\alpha"] & 0\\
            G_0 \arrow[r, "g_0"] & N \arrow[r] & 0
        \end{tikzcd}
        $$
        Note that $g_0 \circ (\tilde{\alpha}_0 \circ f_1) = \alpha \circ f_0 \circ f_1 = 0$, so there is an induced map $i_1 : F_1 \to \ker g_0$ such that the composition $F_1 \overset{i_1}{\to} \ker g_0 \xhookrightarrow{} G_0$ coincides to the map $\tilde{\alpha}_0 \circ f_1$. Since the map $G_1 \to \im g_1 = \ker g_0$ is surjective, there is an induced map $\tilde{\alpha}_1 : F_1 \to G_1$ such that the composition $F_1 \overset{\tilde{\alpha}_1}{\to} G_1 \twoheadrightarrow \im g_1 = \ker g_0$ is the same to the map $i_1$, because the object $F_1$ is projective. Hence, next diagram commutes.
        $$
        \begin{tikzcd}[column sep = tiny, row sep = tiny]
            F_1 \arrow[rr,"f_1"] \arrow[rd, dashed, "i_1"] \arrow[dd, dashed, "\tilde{\alpha}_1"'] && F_0 \arrow[rr, "f_0"] \arrow[dd, "\tilde{\alpha}_0"] & & M \arrow[rr] \arrow[dd, "\alpha"] & & 0\\
            & \bullet \arrow[rd] \\
            G_1 \arrow[rr, "g_1"'] \arrow[ru] && G_0 \arrow[rr, "g_0"'] & & N \arrow[rr] & & 0
        \end{tikzcd}
        $$
        Here, the bullet denotes $\im g_1 = \ker g_0$. Similarly, we can deduce a commutative diagram
        $$
        \begin{tikzcd}[column sep = tiny, row sep = tiny]
            F_2 \arrow[rr,"f_2"] \arrow[rd, dashed, "i_2"] \arrow[dd, dashed, "\tilde{\alpha}_2"'] && F_1 \arrow[rr, "f_1"] \arrow[dd, "\tilde{\alpha}_1"] && F_0 \arrow[rr, "f_0"] \arrow[dd, "\tilde{\alpha}_0"] & & M \arrow[rr] \arrow[dd, "\alpha"] & & 0\\
            & \bullet \arrow[rd] \\
            G_2 \arrow[rr, "g_2"'] \arrow[ru] && G_1 \arrow[rr, "g_1"'] && G_0 \arrow[rr, "g_0"'] & & N \arrow[rr] & & 0
        \end{tikzcd}
        $$
        where the bullet denotes $\im g_2 = \ker g_1$. By repeating this process, we can obtain the commutative diagram
        $$
        \begin{tikzcd}
            \cdots \arrow[r, "f_3"] & F_2 \arrow[r, "f_2"] \arrow[d, "\tilde{\alpha}_2"] & F_1 \arrow[r, "f_1"] \arrow[d, "\tilde{\alpha}_1"] & F_0 \arrow[r, "f_0"] \arrow[d, "\tilde{\alpha}_0"] & M \arrow[r] \arrow[d, "\alpha"] & 0\\
            \cdots \arrow[r, "g_3"] & G_2 \arrow[r, "g_2"] & G_1 \arrow[r, "g_1"] & G_0 \arrow[r, "g_0"] & N \arrow[r] & 0
        \end{tikzcd}
        $$

        Now, let's show the uniqueness up to chain homotopy. It is enough to show that the lift of the zero map is null-homotopic. Again, let $\tilde{\alpha}_\bullet$ be a lift of the zero map.

        By the commutativity, we can deduce that $g_0 \circ \tilde{\alpha}_0 = 0 \circ f_0 = 0$, so $\tilde{\alpha}_0$ factor through $F_0 \overset{i'_0}{\to} \ker g_0 \xhookrightarrow{} G_0$ for some $i'_0$. By the property of projective module, the map $i'_0$ lifts to the map $s_0:F_0 \to G_1$ since the map $G_1 \to \im g_1 = \ker g_0$ is surjective. Thus, we have a commutative diagram.
        $$
        \begin{tikzcd}[column sep = tiny, row sep = tiny]
            && F_0 \arrow[rr, "f_0"] \arrow[dd, "\tilde{\alpha}_0"] \arrow[dl, dashed, "i'_0"] \arrow[ddll, dashed, bend right, "s_0"'] && M \arrow[dd, "0"] \arrow[rr] && 0\\
            & \bullet \arrow[dr] \\
            G_1 \arrow[ur] \arrow[rr, "g_1"'] && G_0 \arrow[rr, "g_0"'] && N \arrow[rr] && 0
        \end{tikzcd}
        $$
        where the bullet denotes $\im g_1 = \ker g_0$. By the diagram chasing, we can conclude that $\tilde{\alpha}_0 = g_1 \circ s_0$. Also, by a diagram chasing, we can deduce that $g_1 \circ (\tilde{\alpha}_1 - s_0 \circ f_1) = 0$, so the map $\tilde{\alpha}$ factor through $F_1 \overset{i'_1}{\to} \ker g_1 \xhookrightarrow{} G_1$ for some $i'_1$. Again, this map lifts to $s_1:F_1 \to G_2$. Therefore, we have a diagram
        $$
        \begin{tikzcd}[column sep = tiny, row sep = tiny]
            && F_1 \arrow[rr, "f_1"] \arrow[dd, "\tilde{\alpha}_1"] \arrow[dl, dashed, "i'_1"] \arrow[ddll, dashed, bend right, "s_1"'] && F_0 \arrow[dd, "\tilde{\alpha}_0"] \arrow[ddll, "s_0"{description}] \arrow[rr, "f_0"] && M \arrow[dd, "0"] \arrow[rr] && 0\\
            & \bullet \arrow[dr] \\
            G_2 \arrow[ur] \arrow[rr, "g_2"'] && G_1 \arrow[rr, "g_1"'] && G_0 \arrow[rr, "g_0"'] && N \arrow[rr] && 0
        \end{tikzcd}
        $$
        where the bullet denotes $\im g_2 = \ker g_1$. Simple diagram chasing proves that $\tilde{\alpha}_1 - s_0 \circ f_1 = g_2 \circ s_1$, or equivalently, $\tilde{\alpha}_1 = s_0 \circ f_1 + g_2 \circ s_1$ holds. Similarly, we have a diagram
        $$
        \begin{tikzcd}[column sep = tiny, row sep = tiny]
            && F_2 \arrow[rr, "f_2"] \arrow[dd, "\tilde{\alpha}_2"] \arrow[dl, dashed, "i'_2"] \arrow[ddll, dashed, bend right, "s_2"'] && F_1 \arrow[rr, "f_1"] \arrow[dd, "\tilde{\alpha}_1"] \arrow[ddll, "s_1"{description}] && F_0 \arrow[dd, "\tilde{\alpha}_0"] \arrow[ddll, "s_0"{description}] \arrow[rr, "f_0"] && M \arrow[dd, "0"] \arrow[rr] && 0\\
            & \bullet \arrow[dr] \\
            G_3 \arrow[ur] \arrow[rr, "g_3"'] && G_2 \arrow[rr, "g_2"'] && G_1 \arrow[rr, "g_1"'] && G_0 \arrow[rr, "g_0"'] && N \arrow[rr] && 0
        \end{tikzcd}
        $$
        where the bullet denotes $\im g_3 = \ker g_2$, and the equality $\tilde{\alpha}_2 = s_1 \circ f_2 + g_3 \circ s_2$ holds. By repeating this process, we can obtain the diagram
        $$
        \begin{tikzcd}
            \cdots \arrow[r, "f_3"] & F_2 \arrow[r, "f_2"] \arrow[d, "\tilde{\alpha}_2"] \arrow[dl, "s_2"{description}] & F_1 \arrow[r, "f_1"] \arrow[d, "\tilde{\alpha}_1"] \arrow[dl, "s_1"{description}] & F_0 \arrow[r, "f_0"] \arrow[d, "\tilde{\alpha}_0"] \arrow[dl, "s_0"{description}] & M \arrow[r] \arrow[d, "0"] & 0\\
            \cdots \arrow[r, "g_3"'] & G_2 \arrow[r, "g_2"'] & G_1 \arrow[r, "g_1"'] & G_0 \arrow[r, "g_0"'] & N \arrow[r] & 0
        \end{tikzcd}
        $$
        such that $\tilde{\alpha}_n = s_{n-1} \circ f_n + g_{n+1} \circ s_n$ holds, i.e., $\tilde{\alpha}_\bullet$ is null-homotopic. This completes the proof.
    \end{proof}

    \begin{proof}[Solution of Problem 6]
        \begin{enumerate}
            \item Let
            $$
            \begin{tikzcd}
                \cdots \arrow[r, "f_3"] & F_2 \arrow[r, "f_2"] & F_1 \arrow[r, "f_1"] & F_0 \arrow[r, "f_0"] & M \arrow[r] & 0
            \end{tikzcd}
            $$
            be a free resolution, then $f^2 = 0$. Since $- \otimes_A N$ is a functor, we have $(f \otimes \id)^2 = 0$, i.e., the sequence
            $$
            \begin{tikzcd}
                \cdots \arrow[r, "f_3 \otimes \id"] & F_2 \otimes_A N \arrow[r, "f_2 \otimes \id"] & F_1 \otimes_A N \arrow[r, "f_1 \otimes \id"] & F_0 \otimes_A N
            \end{tikzcd}
            $$
            is a homological complex.

            \item Let $F_\bullet \to M$ and $G_\bullet \to M$ are free resolutions. Then, by the proof of \text{Problem 5}, the map $\id:M \to M$ lifts to $\tilde{\alpha}:F_\bullet \to G_\bullet$, and by considering the reversed situation, the identity map lifts to $\widehat{\alpha}:G_\bullet \to F_\bullet$. It is easy that the identity map $\id:F_\bullet \to F_\bullet$ is a lift of $\id_M$, so by the proof of \textbf{Problem 5}, we have a chain homotopy $\widehat{\alpha}_\bullet \circ \tilde{\alpha}_\bullet \simeq \id$. Similarly, $\tilde{\alpha}_\bullet \circ \widehat{\alpha}_\bullet \simeq \id$ holds. Hence, two complexes $F_\bullet$ and $G_\bullet$ have the same homotopy type. Now, we need a lemma.
            
            \begin{Lemma}\label{chain homotopic maps induces the same map of homology}
                Let $(C_\bullet, d_\bullet)$ and $(D_\bullet, d'_\bullet)$ are homological complexes, and let $f_\bullet, g_\bullet : C_\bullet \to D_\bullet$ are chain maps. If $f_\bullet \simeq g_\bullet$ holds, then the induced maps $(f_\bullet)_*, (g_\bullet)_* : H_\bullet(C_\bullet) \to H_\bullet(D_\bullet)$ are the same map. In particular, two homotopically equivalent homological complexes gives the same homology.
            \end{Lemma}

            \begin{proof}[Proof of Lemma~\ref{chain homotopic maps induces the same map of homology}]
                Note that any chain map naturally induces a map of homologies as like $(f_\bullet)_\ast : H_n(C_\bullet) \to H_n(D_\bullet), a+\im d_{n+1} \mapsto f_n(a)+\im d'_{n+1}$. This is well-defined since for any $a \in \ker d_n$, the equality $d'_n(f_n(a)) = f_{n-1}(d_n(a)) = 0$ holds, and if $a+\im d_{n+1} = a' + \im d_{n+1}$, then $f_n(a) + \im d'_{n+1} = f_n(a') + f_n(a-a') + \im d'_{n+1} = f_n(a') + \im d'_{n+1}$ since $f_n(a-a') \in \im (f_n \circ d_{n+1}) = \im (d'_{n+1} \circ f_{n-1}) \subseteq \im d'_{n+1}$ holds.

                The remainder is to show that for any $a+\im d_{n+1} \in H_n(C_\bullet)$, we have $f_n(a) + \im d'_{n+1} = g_n(a) + \im d'_{n+1}$ when there is a chain homotopy $s_\bullet : C_\bullet \to D_{\bullet+1}$ such that $f-g = ds+sd$. However, we have $f_n(a) - g_n(a) = (d'_{n+1}s_n + s_{n-1}d_n) (a) = d'_{n+1} s_n (a) + 0 \in \im d'_{n+1}$. This completes the proof of the lemma.
            \end{proof}

            Since $- \otimes_A N$ is a functor, we can easily verify that the tensoring conserves the homotopic equivalence. Therefore, $\Tor$ does not depend on the choice of free resolution, up to isomorphism.

            \item Since the tensoring is right-exact, this send an exact sequence
            $$ F_1 \longrightarrow F_0 \longrightarrow M \longrightarrow 0$$
            to an exact sequence
            $$ F_1 \otimes_A N \longrightarrow F_0 \otimes_A N \longrightarrow M \otimes_A N \longrightarrow 0$$
            Hence, $\Tor_0^A(M,N) = \coker (f_1 \otimes \id) \cong M \otimes_A N$. This completes the proof.
        \end{enumerate}
    \end{proof}

    \begin{proof}[Solution of Problem 7]
        First, let's show that the first condition implies the second condition. Let $G_\bullet \to N$ be a free resolution of $N$, then $\Tor$ is the homology group of the complex $M \otimes_A G_\bullet$. However, the functor $M \otimes_A -$ is exact since $M$ is flat, i.e., the complex
        $$ \cdots \longrightarrow M \otimes_A G_1 \longrightarrow M \otimes_A G_0 \longrightarrow M \otimes_A N \longrightarrow 0 \longrightarrow \cdots $$
        is exact. Thus, we can conclude that $\Tor_n^A(M,N) = 0$ for all $n \ge 1$.

        Second, the condition 2 implies that 3 is clear.

        Finally, let's show that the third condition implies the first one. Let's consider a short exact sequence
        $$ 0 \longrightarrow P' \longrightarrow P \longrightarrow P'' \longrightarrow 0 $$
        Then, the $\Tor$-exact sequence gives an exact sequence
        $$ \Tor_1^A(P'', M) \longrightarrow P' \otimes_A M \longrightarrow P \otimes_A M \longrightarrow P'' \otimes_A M \longrightarrow 0 $$
        However, $\Tor_1^A(P'', M) \cong \Tor_1^A(M, P'') = 0$ holds, so the sequence becomes
        $$ 0 \longrightarrow P' \otimes_A M \longrightarrow P \otimes_A M \longrightarrow P'' \otimes_A M \longrightarrow 0 $$
        Therefore, $M$ is a flat $A$-module, and this completes the proof.
    \end{proof}
    
\end{document}