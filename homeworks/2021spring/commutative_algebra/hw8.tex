\documentclass[titlepage]{scrartcl}
\usepackage{../../styles/style}

\newcommand\authorname{Jeongwoo Park}   % Name of the student
\newcommand\email{jeongwoo-math@kaist.ac.kr}
\newcommand\site{jeongwoo-math.github.io}

\newcommand\university{Korea Advanced Institute of Science and Technology} % Name of the university
\newcommand\department{Department of Mathematical Sciences} % Name of the department
\newcommand\studentid{20190262} % Student ID

\title{
    \articlename
    }
\author{\normalsize
    Last Update : \today\\[1.5em]
    \par\noindent\rule{\textwidth}{0.1mm}\\[2em]
    \textbf{\authorname}\footnote{\; \makebox[2em]{email} \email \newline \makebox[3em]{site} \url{\site}}\\[1em] \normalsize
    \\ \normalsize
    \department,\\ \normalsize
    \university\\[2em]
    \line(1,0){100}\\[2em] \normalsize
    \begin{tcolorbox}[boxrule=0.3mm, colback = white, colframe = black]
        \textbf{Abstract}. \abstract
    \end{tcolorbox}
}
\date{}

\newcommand\homeworknumber{8}

\usepackage{glossaries-extra}

\newglossaryentry{ring map of finite-type}
{
  name={Ring map of finite-type},
  description={A ring map $f:A \to B$ is \emph{finite-type} if and only if $B$ is a finite-type $A$-algebra, where the algebra structure is induced by the ring map $f$.},
  category=symbol
}

\newglossaryentry{embedding-dimension}
{
  name={$\edim$},
  description={embedding-dimension of a local ring},
  category=symbol
}

\begin{document}
    \maketitle

    \printunsrtglossary[title={Terminologies}]

    \newpage

    \section{Solutions}

    \subsection{Problem 1}

        \begin{enumerate}
            \item We have to define a ($A/M$)-module structure on $M/M^2$ making the diagram
            $$
            \begin{tikzcd}%[column sep = ]
                (A/M) \times \left(M/M^2 \right) \arrow[r, "\cdot"] & M/M^2 \\
                A \times M \arrow[u, "\pi_M \times \pi_{M^2}"] \arrow[r, "\cdot"] & M \arrow[u, "\pi_M"']
            \end{tikzcd}
            $$
            commutes where $\pi_M : A \to A/M$ and $\pi_{M^2} : M \to M/M^2$ are natural projections. We define the scalar multiplication as $(a+M) \cdot (m+M^2) = am + M^2$ where $a \in A, m \in M$, then it is well-defined as a function, because if $a+M = a'+M$ and $m+M^2 = m'+M^2$, then $am - a'm' = a \cdot \left( m-m' \right) + (a-a') \cdot m' \in M^2 $ holds. Clearly, this makes the diagram above commutes, because
            $$
            \begin{tikzcd}%[column sep = ]
                \left( a+M, m+M^2 \right) \arrow[r,mapsto] & am+M^2 \\
                (a, m) \arrow[r, mapsto] \arrow[u, mapsto] & am \arrow[u, mapsto]
            \end{tikzcd}
            $$
            holds. Also, it is not too hard to show that $M/M^2$ forms am $(A/M)$-module structure with this scalar multiplication, because it satisfies

            \begin{enumerate}
                \item[unitality] $(1+M) \cdot (m+M^2) = m+M^2$
                \item[left-additivity] $\left( \left( a+M \right) + \left( b+M \right) \right) \cdot \left(m+M^2 \right) = am + bm + M = \left( a+M \right) \cdot \left( m+M^2 \right) + (b+M) \cdot \left( m + M^2 \right)$
                \item[right-additivity] $(a+M) \cdot \left( \left( m+M^2 \right) + \left( n+M^2 \right) \right) = am + an + M^2 = (a+M) \cdot \left( m+M^2 \right) + (a+M) \cdot \left( n+M^2 \right)$
                \item[homogeneity of degree $1$] $(a+M) \cdot \left( (b+M) \cdot \left( m+M^2 \right) \right) = abm + M^2 = \left( (a+M) (b+M) \right) \cdot \left( m+M^2 \right)$
            \end{enumerate}

            hold. Hence, the $A$-module structure on $M$ induces the natural ($A/M$)-module structure on $M/M^2$.

            \item Since $A$ is Noetherian, there are finitely many $m_i \in M$ such that $M = \sum_{i=1}^{n} A \cdot m_i$. It is enough to show that $M/M^2$ is generated by $m_i + M^2$ as an ($A/M$)-module, because every finitely generated vector space is of the finite-dimension. Let $m+M^2 \in M^2$, then there are $a_i$'s such that $m = \sum_{i=1}^n a_i m_i$, and this implies that $m+M^2 = \sum_{i} (a_i + M) \cdot (m_i + M^2)$. Hence, the module $M/M^2$ is generated by $m_i + M^2$, and this completes the proof. \qed
        \end{enumerate}
    
    \subsection{Problem 2}

    \begin{enumerate}
        \item Because $A$ is Artinian as a $k$-module, it must be a finite $k$-algebra, in particular, it is a finitely-type $k$-algebra. One can consider a composition of maps of finite-type
        $$ k \longrightarrow A \longrightarrow A/M $$
        Clearly, the composition is finite-type. By the weak nullstellensatz, the inclusion $k \to A/M$ is a finite extension of fields, because $A/M$ is a field. This proves the statement.

        \item For a trivial example, one can consider a non-trivial finite field-extension $K/k$. By definition, $\dim_k K < \infty$ holds, so $K$ is an Artinian $k$-algebra. Also, $K$ is local because it is a field, and $[K/0:k] = [K:k] > 1$ holds because we assumed the extension is non-trivial. Of course, there is such extension, for example, $\mathbb{C}/\mathbb{R}$.
        
        If necessary, for a non-trivial example, one can consider $k = \mathbb{R}$ and $A = \mathbb{C}[x]/\left\langle x \right\rangle^2$. The complex dimension of $A$ is $2$, hence the real dimension must be $4$, in particular, $A$ is an Artinian $\mathbb{R}$-algebra. By the correspondence theorem, the unique maximal ideal of $\mathbb{C}[x]/\left\langle x \right\rangle^2$ is $\left\langle x \right\rangle/\left\langle x \right\rangle^2$, because $\left\langle x \right\rangle$ is the radical of $\left\langle x \right\rangle^2$, i.e., the unique maximal ideal containing $\left\langle x \right\rangle^2$ is $\left\langle x \right\rangle$. Hence, $A$ is an Artinian local $\mathbb{R}$-algebra with the residue field
        $$ \frac{\mathbb{C}[x]/\left\langle x \right\rangle^2}{\left\langle x \right\rangle/\left\langle x \right\rangle^2} \cong \frac{\mathbb{C}[x]}{\left\langle x \right\rangle} \cong \mathbb{C} $$
        Therefore, the extension degree $[A/M:k]$ is $2$, and this is an example we want to find. \qed
    \end{enumerate}

    \subsection{Problem 3}

    Let's consider a lemma.

    \begin{Lemma}
        Let $A$ be a $k$-algebra which is a Noetherian ring, and $\mathfrak{m} \trianglelefteq A$ be a maximal ideal such that the composition $k \to A \to A/\mathfrak{m}$ is a finite extension of fields. If a radical of an ideal $\mathfrak{a} \trianglelefteq A$ is $\mathfrak{m}$, then the quotient ring $A/\mathfrak{a}$ is an Artinian local $k$-algebra.
    \end{Lemma}

    \begin{proof}
        Since the radical of $\mathfrak{a}$ is $\mathfrak{m}$, the unique prime ideal above $\mathfrak{a}$ is $\mathfrak{m}$, in particular, the ring $A/\mathfrak{a}$ is local by the correspondence theorem.
        
        The remainder is to show that $\dim_k \left( A/\mathfrak{a} \right) < \infty$. Since the map $k \to \kappa$ is a finite extension of fields, where $\kappa$ is a residue field $A/\mathfrak{m}$, it is enough to show that $\dim_{\kappa} \left( A/\mathfrak{a} \right)$ is finite. Note that there is a natural number $N$ such that $\mathfrak{m}^N \subseteq \mathfrak{a}$, because $\sqrt{a} = \mathfrak{m}$ holds and $A$ is a Noetherian ring. If one can show that $\dim_{\kappa} \left(A/\mathfrak{m}^N \right) < \infty$, then we are done, because an inequality $\dim_\kappa (A/\mathfrak{a}) \le \dim_{\kappa} \left(A/\mathfrak{m}^N \right)$ holds.

        One can consider a chain of linear subspaces
        $$ A/\mathfrak{m}^N \supseteq \mathfrak{m}/\mathfrak{m}^N \supseteq \cdots \supseteq \mathfrak{m}^N/\mathfrak{m}^N = 0 $$
        Since the dimension of $A/\mathfrak{m}^N$ is
        $$ \sum_{i=0}^{N-1} \dim_\kappa \left( \frac{\mathfrak{m}^i/\mathfrak{m}^N}{\mathfrak{m}^{i+1}/\mathfrak{m}^N} \right) = \sum_{i=0}^{N-1} \dim_\kappa \left( \frac{\mathfrak{m}^i}{\mathfrak{m}^{i+1}}\right) $$
        it is enough to show that each quotient $\mathfrak{m}^i/\mathfrak{m}^{i+1}$ is of the finite-dimension. Instead of showing the finite-dimensionality, we can show the finite-generation of that modules. Since $A$ is Noetherian as a ring, each module $\mathfrak{m}^i$ is finitely-generated. Similar to the \textbf{Problem~1}, one can easily show that the quotient $\mathfrak{m}^i/\mathfrak{m}^{i+1}$ is finitely-generated $\kappa$-module. This completes the proof.
    \end{proof}

    Note that $\sqrt{\left\langle t^m \right\rangle} = \left\langle t \right\rangle$ and $\sqrt{\left\langle x^2, y^2 \right\rangle} = \sqrt{\left\langle x^3, y^4 \right\rangle} = \sqrt{\left\langle x^3, x^2 y^2, y^4 \right\rangle} = \left\langle x,y \right\rangle$ hold, and the compositions $k \to k[t] \to k[t]/\left\langle t \right\rangle$ and $k \to k[x,y] \to k[x,y]/\left\langle x,y \right\rangle$ are isomorphisms. Of course, every polynomial ring over a field is Noetherian by the Hilbert basis theorem. By applying the lemma, one can conclude that the $k$-algebras in the problem are all Artinian local $k$-algebras. The remainder is to calculate embedding-dimensions of each rings. We need a lemma.

    \begin{Lemma}
        Let $A$ be a $k$-algebra, and let $\mathfrak{m} \trianglelefteq A$ is an ideal such that the composition $k \to A \to A/\mathfrak{m}$ is an isomorphism. Moreover, assume that there are $k$-linearly independent generators $\left\{ m_i \right\}_{1 \le i \le e}$ of $\mathfrak{m}$ such that $\mathfrak{m}^2 \cap \sum_i k \cdot m_i = 0$. For any ideal $\mathfrak{a} \trianglelefteq A$ satisfying $a \subseteq \mathfrak{m}^2$ and $\sqrt{a} = \mathfrak{m}$, the embedding-dimension of a local ring $A/\mathfrak{a}$ is $e$.
    \end{Lemma}

    \begin{proof}
        Note that the maximal ideal of $A/\mathfrak{a}$ is $\mathfrak{m}/\mathfrak{a}$ by the correspondence theorem and the fact that $\sqrt{a} = \mathfrak{m}$, so one an deduce an equality
        $$ \edim\left( A / \mathfrak{a} \right) = \dim_{\frac{A/\mathfrak{a}}{\mathfrak{m}/\mathfrak{a}}} \left( \frac{\mathfrak{m}/\mathfrak{a}}{\mathfrak{m}^2/\mathfrak{a}} \right) = \dim_\kappa \left( \mathfrak{m}/\mathfrak{m}^2 \right) $$
        holds, where $\kappa$ denotes a residue field $A/\mathfrak{m}$, and the second equality can be deduced from the natural isomorphism $\frac{A/\mathfrak{a}}{\mathfrak{m}/\mathfrak{a}} \cong A/\mathfrak{m}$. Note that $m_i + \mathfrak{m}^2$ generates $\mathfrak{m}/\mathfrak{m}^2$ as a $\kappa$-module, so the remainder is to show that it is $\kappa$-linearly independent. It is not too hard to show that the $k$-vector space structure and $\kappa$-vector space structure on $\mathfrak{m}/\mathfrak{m}^2$ are equivalent over the isomorphism $k \to A \to A/\mathfrak{m}$, so it is enough to show that $m_i + \mathfrak{m}^2$ are $k$-linearly independent. Let's assume that $\sum_i a_i \left( m_i + \mathfrak{m}^2 \right) = 0$, then $\sum_i a_i m_i \in \mathfrak{m}^2 \cap \sum_i k \cdot m_i = 0 $ holds. Since $m_i$'s are $k$-linearly independent, $a_i=0$ for every $i$, and this proves the linearly independence. This completes the proof.
    \end{proof}

    It is not too hard to show that the situations in the problem satisfy the assumptions of the lemma above, with the obvious generators; $t$ for $\left\langle t \right\rangle$, and $x,y$ for $\left\langle x,y \right\rangle$. Hence, one can conclude that
    $$ \edim\left( \frac{k[t]}{\left\langle t^m \right\rangle} \right) = 1 $$
    and
    $$ \edim \left( \frac{k[x,y]}{\left\langle x^2,y^2 \right\rangle} \right) = \edim \left( \frac{k[x,y]}{\left\langle x^3,y^4 \right\rangle} \right) = \edim \left( \frac{k[x,y]}{\left\langle x^3, x^2 y^2, y^4 \right\rangle} \right) = 2 $$
    This is the end of the solution. \qed

    \subsection{Problem 4}

    Let's use a notation $\mathfrak{m}$, instead of $M$. It is enough to show that $A$ is generated by $e$ elements as a $k$-algebra, by the next lemma.

    \begin{Lemma}
        Let $B$ be an $A$-algebra generated by $a_i$'s, i.e., $B = A[a_i]_{i \in I}$. Then, there is an ideal $I$ of a polynomial algebra $A[x_i]_{i \in I}$ such that $B \cong A[x_i]_i / I$.
    \end{Lemma}


    \begin{proof}
        One can consider an $A$-algebra epimorphism $\phi : A[x_i]_i \to A[a_i]_i; x_i \mapsto a_i$. Take $I$ as a kernel of $\phi$, then one can conclude that $A[a_i]_i \cong A[x_i]_i / I$, by the first isomorphism theorem. This completes the proof.
    \end{proof}

    Let $m_1+\mathfrak{m}^2, \cdots, m_e+\mathfrak{m}^2$ be a $\kappa$-basis of $\mathfrak{m}/\mathfrak{m}^2$, where $\kappa$ is a residue field $A/\mathfrak{m}$. Since the composition $k \to A \to A/\mathfrak{m}$ is an isomorphism, and the $k$-vector space structure and $\kappa$-vector space structure are compatible, $m_i + \mathfrak{m}^2$'s can be viewed as a $k$-basis of $\mathfrak{m}/\mathfrak{m}^2$. It is not too hard to show that $\left\{ m_{i_1} \cdots m_{i_n} + \mathfrak{m}^{n+1} \right\}_{(1 \le i_j \le e}$ generates $\mathfrak{m}^n/\mathfrak{m}^{n+1}$ as a $k$-module.

    Let $a \in A$, then there is $c \in k$ such that $a-c \in \mathfrak{m}$, because the composition $k \to A \to A/\mathrm{m}$ is an isomorphism. Also, there are $c_i \in k$ such that $a-c - \sum_i c_i \cdot m_i \in \mathfrak{m}^2$, since $m_i + \mathfrak{m}^2$ generates $\mathfrak{m}/\mathfrak{m}^2$. Again, because $\left\{ m_{i_1} m_{i_2} + \mathfrak{m}^3 \right\}_{i_1,i_2}$ generates $\mathfrak{m}^2 / \mathfrak{m}^3$, there are scalars $c_{i_1, i_2} \in k$ such that $a - c - \sum_i c_i \cdot m_i - \sum_{i_1, i_2} c_{i_1, i_2} \cdot m_{i_1} m_{i_2} \in \mathfrak{m}^3$. By repeating this argument until $\mathfrak{m}^{n+1} = 0$, we can conclude that
    $$ a = c + \left(\sum_{i_1} c_{i_1} \cdot m_{i_1}\right) + \cdots + \left(\sum_{i_1, \cdots, i_n} c_{i_1, \cdots, i_n} \cdot m_{i_1} \cdots m_{i_n}\right) \in k[m_i]_i $$
    holds, for some $c, c_{i_1}, \cdots, c_{i_1, \cdots, i_n} \in k$. This implies that $A \subseteq k[m_i]_i$, so one can deduce that $A = k[m_i]_i$. Therefore, $A$ is generated by $e$ elements, and this completes the proof. \qed

    \subsection{Problem 5}

    \begin{enumerate}
        \item Let's assume that $A \cong k[t]/I$, then there is exactly one maximal ideal containing $I$, because $k[t]/I$ is local. Since $I$ is a PID that is not a field, $I$ is non-zero and generated by $\prod_{i=1}^{n} p_i^{e_i}$ where $p_i$'s are prime and $e_i$'s are positive integers. One can easily show that $\sqrt{I} = \bigcap_i \left\langle p_i \right\rangle$. Since there are at most one maximal ideal containing $I$, and $\dim k[x] = 1$ because $k[t]$ is a PID, the number $n$ must be $1$. Hence, $I = \left\langle p_1^{e_1} \right\rangle$ holds. Since we assumed that the composition
        $$ k \longrightarrow k[t] \longrightarrow \frac{k[t]/I}{\left\langle p_1 \right\rangle/I} \cong k[t]/\left\langle p_1 \right\rangle $$
        is an isomorphism, one can deduce that $t + \left\langle p_1 \right\rangle = c + \left\langle p_1 \right\rangle$ for some $c \in k$. This means that $t-c \in \left\langle p_1 \right\rangle$, i.e., $\left\langle p_1 \right\rangle = \left\langle t-c \right\rangle$ holds, because $\left\langle t-c \right\rangle$ is maximal. Therefore, $I = \left\langle t-c \right\rangle^{e_{1}}$, and we can conclude that
        $$ A \cong \frac{k[t]}{I} = \frac{k[t]}{\left\langle t-c \right\rangle^{e_1}} \cong \frac{k[t]}{\left\langle t^{e_1} \right\rangle} $$
        where the last isomorphism can be shown by using an isomorphism $k[t] \to k[t]; t \mapsto t+c$. Of course, $A$ is complete intersection by definition. This completes the proof.

        \item By \textbf{Problem~3}, the ring $k[x,y]/\left\langle x^2, y^2 \right\rangle$ is of embedding-dimension $2$, and it is of the complete intersection by definition.
        
        \item Let's consider algebraically closed $k$, and let $A = k[x,y]/\left\langle x,y \right\rangle^2$. By the method I used to solve \textbf{Problem~3}, one can show that $\edim \left( \frac{k[x,y]}{\left\langle x,y \right\rangle^2} \right) = 2$. The remainder is to show that it is not a complete intersection. Suppose that it is isomorphic to $\frac{k[x,y]}{I}$, where $I = \left\langle f,g \right\rangle$ for some $f,g \in k[x,y]$. Note that there is exactly one maximal ideal containing $I$ because the quotient ring is local, so $\sqrt{I} = \left\langle x-a, y-b \right\rangle$ holds for some $a,b \in k$. This is because every maximal ideal of $k[x,y]$ is of the that form, since $k$ is algebraically closed. Note that the square of the maximal ideal of $A$ is zero, so $I \supseteq \left\langle x-a, y-b \right\rangle^2$ holds. Note that $\bar 1, \bar x, \bar y$ forms a $k$-basis of $A$, $\dim_k A = \dim_k \left( \frac{k[x,y]}{I} \right) = 3$. Similarly, $\dim_k \left( \frac{k[x,y]}{\left\langle x-a, y-b \right\rangle^2} \right) = 3$ holds, but since $\frac{k[x,y]}{I}$ is a quotient vector space of $\frac{k[x,y]}{\left\langle x-a,y-b \right\rangle^2}$, one can conclude that $I = \left\langle x-a, y-b \right\rangle^2$. However, $I$ can't be generated by two elements, because if it was, then the ($\frac{A}{\left\langle x-a,y-b \right\rangle}$)-vector space $\frac{I}{\left\langle x-a, y-b \right\rangle \cdot I}$ is generated by two elements. However, $(x-a)^2, (x-a)(y-b), (y-b)^2$ forms a $k$-basis (or equivalently, ($\frac{A}{\left\langle x-a,y-b \right\rangle}$)-basis) of $\frac{I}{\left\langle x-a, y-b \right\rangle \cdot I}$, so it can't be generated by two elements. Hence, $A$ is not a complete intersection, and this completes the proof. \qed
    \end{enumerate}

\end{document}