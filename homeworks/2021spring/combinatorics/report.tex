\documentclass[titlepage]{scrartcl}
\usepackage{../../styles/style}

\newcommand\authorname{Jeongwoo Park}   % Name of the student
\newcommand\email{jeongwoo-math@kaist.ac.kr}
\newcommand\site{jeongwoo-math.github.io}

\newcommand\university{Korea Advanced Institute of Science and Technology} % Name of the university
\newcommand\department{Department of Mathematical Sciences} % Name of the department
\newcommand\studentid{20190262} % Student ID

\title{
    \articlename
    }
\author{\normalsize
    Last Update : \today\\[1.5em]
    \par\noindent\rule{\textwidth}{0.1mm}\\[2em]
    \textbf{\authorname}\footnote{\; \makebox[2em]{email} \email \newline \makebox[3em]{site} \url{\site}}\\[1em] \normalsize
    \\ \normalsize
    \department,\\ \normalsize
    \university\\[2em]
    \line(1,0){100}\\[2em] \normalsize
    \begin{tcolorbox}[boxrule=0.3mm, colback = white, colframe = black]
        \textbf{Abstract}. \abstract
    \end{tcolorbox}
}
\date{}

\usepackage[
backend=biber,
style=alphabetic,
sorting=ynt
]{biblatex}
\setcounter{biburlnumpenalty}{10000}
\setcounter{biburllcpenalty}{7000}
\setcounter{biburlucpenalty}{8000}

\addbibresource{references.bib}

\DeclareMathAlphabet{\mathcal}{OMS}{cmsy}{m}{n}

\theoremstyle{plain}
\newtheorem{theorem}[subsection]{Theorem}
\newtheorem{lemma}[subsection]{Lemma}

\theoremstyle{definition}
\newtheorem{definition}[subsection]{Definition}

\begin{document}
    \maketitle

    \begin{abstract}
        \section*{Abstract}

        One of the most basic and powerful tool in combinatorics is \emph{counting case by case}. This article includes a polynomial method for partitioning sets, which can be used to apply case-by-case-counting method. Also, I'll introduce its applications, like the \href{https://en.wikipedia.org/wiki/Szemer%C3%A9di%E2%80%93Trotter_theorem}{\uwave{Szemer\'edi--Trotter theorem}}.
    \end{abstract}

    \tableofcontents

    \newpage

    \section{Brief Introduction}

    The \href{https://en.wikipedia.org/wiki/Ham_sandwich_theorem}{\uwave{ham sandwich theorem}} states that, for any $d$ bounded open subsets of a Euclidean space $\mathbb{R}^d$, there is a \emph{hyperplane} bisecting each of them, in the sense of the \emph{Lebesgue measure}. One can consider \emph{hypersurfaces} instead of hyperplanes. In this case, we can bisect \emph{more than $d$} bounded open subsets of $\mathbb{R}^d$. Also, there are discrete analogues of them, by replacing the Lebesgue measure to the \emph{counting measure}, and bounded open subsets to finite subsets. We can use this to make a \emph{nice division} of a finite set, which can be applied to prove Szemer\'edi--Trotter theorem.


    \section{The Ham Sandwich theorems}

    \begin{theorem}[Borsuk--Ulam theorem]
        
    \end{theorem}

    The classical ham sandwich theorem can be stated as follows.

    \begin{definition}[bisection]
        Let $A$ be a Lebesgue-measurable subset of $\mathbb{R}^d$. We call a hyperplane $H = \left\{ x \in \mathbb{R}^d \,;\, x \cdot v = b \right\}$ \emph{bisects $A$}, if and only if, $m\left( A \cap H^+ \right) = m\left( A \cap H^- \right) = \frac{1}{2} \cdot m(A)$, where $H^+ := \left\{ x \in \mathbb{R}^d \,;\, x \cdot v > b \right\}$ and $H^- := \left\{ x \in \mathbb{R}^d \,;\, x \cdot v < b \right\}$ are two half-spaces.\footnote{The well-definedness is not too hard to show, because $H^+$ and $H^-$ can be considered as the two connected components of $\mathbb{R}^d \setminus H$. However, I prefer the former definition, because it is more fundamental and easy to generalize.}
    \end{definition}

    \begin{theorem}[classical ham sandwich theorem]
        Let $\left( O_i \right)_{1 \le i \le d}$ be a collection of bounded open subsets of $\mathbb{R}^d$. There is a hyperplane $H$ bisecting each of $\left( O_i \right)_{1 \le i \le d}$.
    \end{theorem}
    
    \begin{proof}
        Use Borsuk--Ulam theorem. Standard proofs can be found in {\color{red}(???)}
    \end{proof}

    \begin{theorem}[polynomial ham sandwich theorem]
        
    \end{theorem}

    \begin{theorem}[general ham sandwich theorem, cf. \cite{stone1942}]
        
    \end{theorem}

    \begin{theorem}[discretized classical sandwich theorem]
        
    \end{theorem}

    \begin{theorem}[discretized polynomial sandwich theorem]
        
    \end{theorem}

    \begin{definition}[Veronese embedding]
        
    \end{definition}

    \section{Cell Decomposition}

    \begin{lemma}[cell decomposition]
        
    \end{lemma}

    \begin{theorem}[\cite{thom1965}]
        
    \end{theorem}

    \begin{lemma}[B\'ezout's theorem, cf. \cite{fulton2008}]
        
    \end{lemma}

    \section{Szemer\'edi--Trotter Theorem}

    \begin{theorem}[Szemer\'edi--Trotter theorem]
        
    \end{theorem}

    

    \cite{kaplan2011}
    \cite{munkres2000}
    \cite{thom1965}
    \cite{stone1942}
    \cite{tao2011}
    \cite{yuval2015}

    \newpage

    \printbibliography

\end{document}