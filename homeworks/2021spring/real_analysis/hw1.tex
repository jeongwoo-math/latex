\documentclass{scrartcl}
\usepackage{../../styles/style}

%\darkmode
% https://jangsookim.github.io/lectures/vscode/vscode_lecture0.html

\DeclareMathOperator{\osc}{osc}

\title{
    Real Analysis \\ \Large
    --- Solution of Homework 1 ---
    }
\author{20190262 Jeongwoo Park}
\date{}

\begin{document}
    \maketitle

    \section{Solutions}

    \begin{proof}[Solution of the Exercise 1.4]
        Let $U_{i,j}$ be the $j$th open interval removed at the $i$th stage where $0 \le j \le 2^{i-1}-1$, and let $C_{i,j}$ be the $j$th closed interval at the $i$th stage where $0 \le j \le 2^i-1$. Of course, these intervals are non-empty. Then, we have properties

        \begin{enumerate}
            \item The disjoint union $[0,1] \setminus \widehat{\mathcal{C}} = \coprod_{i \ge 1} \coprod_{j=0}^{2^{i-1}-1} U_{i,j}$ and $C_{i,j} = C_{i+1,2j} \amalg U_{i+1,j} \amalg C_{i+1,2j+1}$ holds.
            
            \item The length or the Lebesgue measure of $U_{i,j}$ is $\ell_i$, and the length of $C_{i,j}$ converges to $0$ as $i$ goes to $\infty$.
        \end{enumerate}

        \begin{enumerate}
            \item[(a)] Since every intervals are measurable, we have
            $$ m(\widehat{\mathcal{C}}) = m \left( [0,1] \setminus \coprod_{i \ge 1} \coprod_{j=0}^{2^{i-1}-1} U_{i,j} \right) = m([0,1]) - \sum_{i \ge 1} \sum_{j=0}^{2^{i-1}-1} \mu(U_{i,j}) = \sum_{i \ge 1} = 1-\sum_{i \ge 1} 2^{i-1} \ell_i $$
            This completes the proof of the identity.

            \item[(b)] Let $x \in \widehat{\mathcal{C}}$, then for each $i \ge 1$, there is $0 \le j_i \le 2^i-1$ such that $x \in C_{i,j_i}$. Note that
            $$ \bigcap_{i \ge 1} C_{i,i_j} = \{x\} $$
            since the diameter of $C_{i,i_j}$ converges to $0$. By the property $1$, we can know that $j_{i+1} \in \{2j_i, 2j_i+1\}$. Now, pick any element $x_i \in U_{i-1,j_i} \subseteq C_{i,j_i}$, then it must converge to $x$. Also, $U_{i,j_i}$ are sub-intervals of $\widehat{\mathcal{C}}$, and their length converges to $0$. This completes the proof.

            \item[(c)] Of course, $\widehat{\mathcal{C}}$ is closed since it is a complement of an open set in the space $[0,1]$. Also, there is no isolated point because if we take $x_i \in C_{i,j_i} \setminus x$ in the proof of the part $b$, then it must converge to $x$. Hence, $\widehat{\mathcal{C}}$ is perfect.
            
            The space $\widehat{\mathcal{C}}$ is interior-empty because the length of $C_{i,j}$ converges to $0$ as $i$ goes to infinity.

            \item[(d)] We need a lemma.
            
            \begin{Lemma}\label{non-empty perfect complete metric space is uncountable}
                 Every non-empty perfect complete metric space is uncountable.
            \end{Lemma}

            \begin{proof}[Proof of Lemma~\ref{non-empty perfect complete metric space is uncountable}]
                Let $(X,d)$ be a non-empty perfect complete metric space. By the Baire category theorem, $(X,d)$ must be a Baire space. Assume that the space $X$ is countable, and let's write $X = \{x_i\}_{i \in \mathbb{N}}$. Of course, $X$ can't be finite because every point in a finite Hausdorff space is isolated. Since there is no isolated point, open sets $U_j := \{x_i\}_{i \ge j}$ is dense. However, the intersection of all $U_j$'s gives an empty set, which is not dense in $X$. This contradicts to the fact that $X$ is a Baire space, and this completes the proof of the lemma.
            \end{proof}

            Since $\widehat{\mathcal{C}}$ is a closed subset of a complete metric space $\mathbb{R}$, it is a complete metric space. Thus, $\widehat{\mathcal{C}}$ is a non-empty perfect complete metric space, so we can deduce that this space is uncountable. This completes the proof.
        \end{enumerate}
    \end{proof}

    \begin{proof}[Solution of Exercise 1.7]
        First, I'll show that $m^*(\delta \cdot E) = \delta_1 \delta_2 \cdots \delta_d \cdot m^*(E)$. Note that $\delta$ induces a bijection
        $$ \{\text{Hypercubes containing } E \} \overset{\delta \cdot \bullet}{\longrightarrow} \{\text{Hypercubes containing } \delta \cdot E \} $$
        Hence, we can conclude that
        \begin{align*}
            m^*(\delta \cdot E) &= \inf \{|\delta \cdot E| \,;\, E \text{ is a hypercube containing } E\} \\
            &= \inf \{ \delta_1 \delta_2 \cdots \delta_d \cdot |E| \,;\, E \text{ is a hypercube containing } E\}\\
            &= \delta_1 \delta_2 \cdots \delta_d \cdot m^*(E)
        \end{align*}
        Now, the remainder is to show that the set $\delta \cdot E$ is measurable. By the definition, for every $\varepsilon>0$, there is an open set $U \supseteq E$ such that $m^*(U \setminus E) < \frac{\varepsilon}{\delta_1 \delta_2 \cdots \delta_d}$. Also, $\delta \cdot U$ is an open set containing $E$, and we have an inequality
        $$ m^*(\delta \cdot U \setminus \delta \cdot E) = m^*(\delta(U \setminus E) = \delta_1 \delta_2 \cdots \delta_d \cdot m^*(U \setminus E) < \varepsilon $$
        Hence, $\delta \cdot E$ must be measurable. This completes the proof.
    \end{proof}

    \begin{proof}[Solution of Exercise 1.24]
        We define sets $A_i := \left\{n \in \mathbb{Z}_{>0} \,;\, 2^i \Vert n \right\}$\footnote{The relation $\Vert$ denotes the \emph{exact divisibility}, i.e., for any prime number $p$ and natural numbers $k$ and $n$, the relation $p^k \Vert n$ means that $p^k \mid n$ but $p^{k+1} \nmid n$.} and $B_i := \mathbb{Q} \cap \left(B_{\frac{1}{i-1}}(e) \setminus B_{\frac{1}{i}}(e)\right)$ where $e$ is any irrational number, and we define $B_{\frac{1}{0}} := \mathbb{R}$. Then, the $A_i$'s (respectively, $B_i$'s) forms a partition of $\mathbb{Z}_{>0}$ (respectively, $\mathbb{Q}$). Also, every $A_i$ and $B_i$ is countably infinite, so we can find a bijection $f_i : A_i \to B_i$. By gluing all $f_i$'s, we can define an bijection $f : \mathbb{Z}_{>0} \to \mathbb{Q}$ satisfying $f|_{A_i} = f_i$. In particular, $|f(n) - e| > \frac{1}{n}$ holds, because there is $i$ such that $A_i \ni n$, and we have $n \ge 2^i$ because $2^i | n$, and so $|f(n)-e| > \frac{1}{i} > \frac{1}{n}$ holds. Hence, $e \notin \left( f(n)-\frac{1}{n}, f(n)+\frac{1}{n} \right)$ for all $n$, and so $e \notin \bigcup_{n \ge 1} \left( f(n)-\frac{1}{n}, f(n)+\frac{1}{n} \right)$. This completes the proof.
    \end{proof}

    \begin{proof}[Solution of Exercise 1.35]
        We need a lemma.
        \begin{Lemma}\label{Vitali}
            Every set $A \subseteq \mathbb{R}$ of positive outer measure has a non-measurable subset.
        \end{Lemma}

        \begin{proof}[Proof of Lemma~\ref{Vitali}]
            We can assume that $A$ is measurable, and moreover bounded because
            $$ \lim_{R \to \infty} m^*([-R, R] \cap A) = m^*(A) $$
            implies that there is $R>0$ such that $m^*([-R, R] \cap A) > 0$, and every subset of $[-R, R] \cap A$ is also a subset of $A$. Assume that $A$ is bounded by $R>0$.

            Let's consider the equivalence relation on $A$ inherited from the equivalence relation on $\mathbb{R}$ defined by the natural $\mathbb{Q}$-action on $\mathbb{R}$, as in the construction of a Vitali set. Pick a set of representatives $A' \subseteq A$, then we can obtain that
            $$ A \subseteq \coprod_{q \in [-2R,2R] \cap \mathbb{Q}} q+A' \subseteq [-3R,3R] $$
            If $A'$ is measurable, then the inequality
            $$ 0 < m(A) \le \sum_{q \in [-2R,2R] \cap \mathbb{Q}} m(A') \le 6R $$
            holds. However, if $m(A')=0$, this contradicts to the left inequality, and if $m(A')>0$, then this contradicts to the right inequality. Thus, $A' \subseteq A$ is non-measurable. This completes the proof.
        \end{proof}

        We consider a continuous function $f : x \mapsto x+c(x)$ where $c$ is the Cantor--Lebesgue function. Then, $f^{-1}([0,1] \setminus \mathcal{C}) = 2 \cdot \left([0,1] \setminus \widehat{\mathcal{C}}\right)$ where $\widehat{\mathcal{C}}$ is the Cantor-like set defined by the sequence $\ell_i = \frac{1}{2 \cdot 3^i}$ as in the \textbf{Exercise 1.4}. By a simple calculation, $m\left( f^{-1}(\mathcal{C}) \right) = 1$. By the lemma above, there is a non-measurable subset $A$ of $f^{-1}\left( \mathcal{C} \right)$, but $f(A) \subseteq \mathcal{C}$ is measurable because it must be a null-set. Now, we consider the composition
        $$ [0,1] \overset{f}{\longrightarrow} [0,2] \overset{\chi_{f(A)}}{\longrightarrow} [0,1] $$
        This composition is not measurable because if we consider an inverse image of the set $\{1\} \subseteq [0,1]$, then we have the non-measurable set $f^{-1}(f(A)) = A$ since $f$ is injective. This completes the proof.
    \end{proof}

    \begin{proof}[Solution of Problem 1.4]
        \begin{enumerate}
            \item[(a)] It is enough to show that the set given in the problem is closed because every closed subset of a compact set is compact. In the other word, it is enough to show that the set
            $$ A_{\varepsilon}^c = \left\{ c \in J \,;\, \osc(f,c) < \varepsilon \right \} $$
            is open. Let $c \in A_{\varepsilon}^c$, then there is a small $\delta>0$ such that $\osc(f,c,r) < \varepsilon$ for all $0<r<\delta$. By definition of the oscillation function, for any $c' \in ]c-\delta, c+\delta[$, we have $\osc(f,c',r') \le \osc(f,c,r) < \varepsilon$ for all $0<r'< \min \left\{|c'-c-\delta|, |c'-c+\delta| \right\}$. Since the function $\osc(f,c, \bullet)$ is non-increasing function, we can conclude that $]c-\delta, c+\delta[ \subseteq A_{\varepsilon}^c$, hence $A_{\varepsilon}^c$ is open.

            \item[(b)] Suppose that the function is bounded by $M>0$, and let $\varepsilon>0$. We can take finitely many disjoint open intervals $I_i$'s satisfying $A_{\frac{\varepsilon}{m(J)}} \subseteq \bigcup_{i=1}^{n} I_i$ and $\sum_{i=1}^{n} m(I_i) < \frac{\varepsilon}{4M}$ since $A_\varepsilon$ is a null-set because it is a subset of a collection of all the discontinuities, and $A_{\varepsilon}$ is compact. For point $c$ outside of $I_i$'s, we define $r_c>0$ as a number such that $\osc(f,c,r_c) < \frac{\varepsilon}{2m(J)}$. Of course, $\{I_i\}_{1 \le i \le n} \cup \{B_{r_c}(c)\}_{c \notin \bigcup_i I_i}$ is an open covering of the compact interval $J$, so there is a countable subcovering. If we define a partition of $J$ as the set of all boundary points of the countable subcovering, then we can conclude that
            $$ U(f,P) - L(f,P) \le \sum_{I \,;\, I \subseteq \bigcup_i I_i} 2M \cdot m(I) + \sum_{I \,;\, I \subsetneq \bigcup_i I_i} \frac{\varepsilon}{2m(J)} \cdot m(I) \le \frac{\varepsilon}{2}+\frac{\varepsilon}{2} = \varepsilon$$
            where $I$ runs over the subinterval of the partition $P$. This completes the proof.

            \item[(c)] Note that the set of all the discontinuities is $\bigcup_{n=1}^{\infty} A_{\frac{1}{n}}$. Let $P$ be a partition of the interval $J$ such that $U(f,P)-L(f,P) < \frac{\varepsilon}{n}$ where $\varepsilon>0$, and $n$ is any positive integer. Then, we can conclude that
            $$ \frac{\varepsilon}{n} > U(f,P)-L(f,P) \ge \sum_{I \,;\, I \cap A_{\frac{1}{m}} \neq \varnothing} \osc \left(f,c \in I \cap A_{\frac{1}{n}} \right) \cdot m(I) \ge \frac{m \left(A_{\frac{1}{n}}\right)}{n} $$
            or equivalently, $m\left( A_{\frac{1}{n}} \right) < \varepsilon$. Since $\varepsilon>0$ is arbitrary, we can conclude that $m\left( A_{\frac{1}{n}} \right) = 0$, hence the set of all the discontinuities is a null-set.
        \end{enumerate}
    \end{proof}
\end{document}