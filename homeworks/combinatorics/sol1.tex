\documentclass[a4paper,11pt]{article}
\usepackage{amsmath,amssymb,amsthm, tikz,titlesec,hyperref, mathrsfs, mathtools}
\usepackage[a4paper,margin=2cm]{geometry}
\usepackage{ulem}
\linespread{1.3}
\newtheorem{claim}{Claim}[section]

%%%%%%%%%%%%%%%%%%%%%%%% My Settings %%%%%%%%%%%%%%%%%%%%%%%%%

\hypersetup{
    colorlinks=true
}

\DeclareMathOperator*{\codim}{codim}
\DeclareMathOperator*{\im}{im}
\DeclareMathOperator*{\rank}{rank}
\DeclareMathOperator*{\Span}{Span}

\newcommand{\dx}{\mathrm{d}x}

% Quotient
\newcommand*{\quo}[2]{ % \newfaktor{#1}{#2} -> #1/#2
  \raisebox{0.8\height}{\ensuremath{#1}} % Numerator
  \mkern-7mu\raisebox{-0.2\height}{\scalebox{2}{$\diagup$}}\mkern-7mu % Slash /
  \raisebox{-0.9\height}{\ensuremath{#2}} % Denominator
}

%%%%%%%%%%%%%%%%%%%%%%%%%%%%%%%%%%%%%%%%%%%%%%%%%%%%%%%%%%%%%%

\newcommand\name{Jeongwoo Park}   % Name of the student
\newcommand\university{KAIST} % Name of the university
\newcommand\department{Mathematical Sciences} % Name of the department
\newcommand\studentid{20190262} % Student ID

\title{KAIST\\2021 MAS575 Combinatorics\\
Homework\bigskip}
\author{\textbf{\Large \name} \\
University: \university\\
Department: \department\\
Student ID: \studentid}
\date{\today}

\begin{document}
\thispagestyle{empty}
\maketitle
\tableofcontents
\titleformat{\section}[frame]{\pagebreak}{\filright
\footnotesize  \enspace \textsf{KAIST --- MAS575 Combinatorics 2021 Spring}\enspace}{6pt}{\Large\bfseries\filcenter}

\section{HW 1.1}  

Let $r_i$ be the $\mathbb{F}_2$-coefficient characteristic (column) vector of the set $R_i$, i.e., $r_i := (\chi_{R_i}(j))_{1 \le j \le n} \in \mathbb{F}_2^n$ where $\chi_\bullet$ denotes the $\mathbb{F}_2$-coefficient \href{https://en.wikipedia.org/wiki/Indicator_function}{\uwave{indicator function}}. Similarly, let $b_i$ be the $\mathbb{F}_2$-coefficient characteristic vector of the set $B_i$. Then, the condition (a) can be interpreted as $r_i \cdot b_i = 1$, and the condition (b) can be interpreted as $r_i \cdot b_j = 0$ for any $i \neq j$. If we define two matrices $R := (r_i(j))_{(j,i)}, B := (b_i(j))_{(j,i)} \in \mathbb{F}_2^{n \times m}$, then, we have an equality
$$ R^\intercal B = (r_i \cdot b_j)_{1 \le i,j \le m} = I_{m \times m} $$
By rank inequalities, we can conclude that
$$ m = \rank (R^\intercal B) \le \rank B \le n $$
because $B$ is a $n$ by $m$ matrix. This completes the proof. \qed

\section{HW 1.2}

Let $(C_i)_{1 \le i \le m}$ be clubs, and let $c_i$ be the $\mathbb{F}_2$-coefficient characteristic vector of the set $C_i$. Then, the first condition can be interpreted as $c_i \cdot c_i = 0$, and the second condition can be interpreted as $c_i \cdot c_j = 1$ for all $i \neq j$. If we define a matrix $C := (c_i(j))_{(j,i)} \in \mathbb{F}_2^{n \times m}$, then we can obtain that
$$ C^\intercal C = (c_i \cdot c_j)_{(i,j)} = (1- \delta_{ij})_{(i,j)} \in \mathbb{F}_2^{m \times m} $$

\begin{claim}\label{rank of 1-I}
     The rank of a matrix $(1-\delta_{ij})_{(i,j)} \in \mathbb{F}_2^{m \times m}$ is $2 \lfloor \frac{m}{2} \rfloor$.
\end{claim}

\begin{proof}[Proof of Claim~\ref{rank of 1-I}]
    First, let's assume that $m$ is an even number, and let $(1-\delta_{ij})_{(ij)} x = 0$ where $x = (x_i)_i \in \mathbb{F}_2^m$. To show that $\rank(1-\delta_{ij})_{(i,j)} = m$, it is enough to show that $x=0$. However, the equation $(1-\delta_{ij})_{(ij)} x = 0$ implies that $\sum_{k=1}^{m} x_k = x_i$ for all $i$. If we sum all these equations, then we have $\sum_{k=1}^{m} x_k = 0$ because $m$ is an even number, and the characteristic of $\mathbb{F}_2$ is $2$. Therefore, $x_i = \sum_{k=1}^m x_k = 0$ for all $i$, i.e., $x=0$.

    Now, let's assume that $m$ is an odd number. Note that the $m-1$ by $m-1$ submatrix of $(1-\delta_{ij})_{(i,j)}$ at the top-left corner is of the rank $m-1$ by the result above, so $\rank (1-\delta_{ij})_{(i,j)} \ge m-1 = 2 \lfloor \frac{m}{2} \rfloor$. The remainder is to show that the matrix is singular. However, $(1-\delta_{ij})_{(i,j)} (1)_{1 \le i \le m} = 0$ holds, and this completes the proof.
\end{proof}

By a rank inequality, we can obtain that
$$ 2 \left \lfloor \frac{m}{2} \right \rfloor = \rank (C^\intercal C) \le \rank C $$
If the inequality $\rank C \le n-1$ holds, then $m \le n-1$ if $n$ is even, and $m \le n$ if $n$ is odd. This is because if $n$ is even, then $2 \lfloor \frac{m}{2} \rfloor \le n-1$ implies that $2 \lfloor \frac{m}{2} \rfloor \le n-2$, or equivalently, $\lfloor \frac{m}{2} \rfloor \le \frac{n-2}{2}$. Since $\frac{m-1}{2} \le \lfloor \frac{m}{2} \rfloor$, the inequality $\frac{m-1}{2} \le \frac{n-2}{2}$ holds, and this implies that $m \le n-1$. The case for odd $n$ can be done by a similar method.

Now, the remainder is to show that $\rank C \le n-1$. For all $x \in \mathbb{F}_2^m$, we have
$$ (Cx)^\intercal(Cx) = x^\intercal C^\intercal C x = x^\intercal (1-\delta_{ij})_{(i,j)} x = x^\intercal \left( \left( \sum_{k=1}^{m} x_k \right) - x_{i} \right)_{1 \le i \le m} = \left( \sum_{k=1}^{m} x_k \right)^2 - \sum_{k=1}^{m} x_k^2 = 0 $$
by the \href{https://en.wikipedia.org/wiki/Freshman%27s_dream}{\uwave{Freshman's dream}}. This implies that $\im C \le H := \left\{ x \in \mathbb{F}_2^m \,;\, x \cdot x = 0 \right\}$. However, $H \lneq \mathbb{F}_2^m$ because $(1, 0, \cdots, 0) \notin H$, i.e., $\dim H \le n-1$. Therefore, $\rank C = \dim \im C \le n-1$, and this completes the proof. \qed

\section{HW 1.3}

Let's write $\mathscr{A} (= \mathcal{A}) = \{A_i\}_{1 \le i \le a}$ and $\mathscr{B} (= \mathcal{B}) = \{B_j\}_{1 \le j \le b}$, and let $a_i$ be the $\mathbb{F}_2$-coefficient characteristic vector of $A_i$, and $b_j$ be the $\mathbb{F}_2$-coefficient characteristic vector of $B_j$. Then, the given condition can be interpreted as $a_i \cdot b_j = 1$ for all $i$ and $j$. If we define two vectors $a'_i, b'_i \in \mathbb{F}_2^{n+1}$ as $a'_i := (a_i,1)$ and $b'_i := (b_i,1)$, then we have $a'_i \cdot b'_i = 0$ for all $i$ and $j$. Let's consider two vector spaces $A := \langle a_i \rangle_i$ and $B := \langle b_j \rangle_j$, then $A \perp B$ hold, in particular, $\dim A + \dim B \le n+1$ holds.

Let's identity $\mathbb{F}_2^n$ as a subspace of $\mathbb{F}_2^{n+1}$ by using a monomorphism $\mathbb{F}_2^n \xhookrightarrow{v \mapsto (v,0)} \mathbb{F}_2^{n+1}$. Then, $A+\mathbb{F}_2^n = \mathbb{F}_2^{n+1}, B+\mathbb{F}_2^n = \mathbb{F}_2^{n+1}$ hold because $\codim_{\mathbb{F}_2^{n+1}} \mathbb{F}_2^n = 1$ and $A+\mathbb{F}_2^n, B+\mathbb{F}_2^n \gneq \mathbb{F}_2^n$. Therefore, we have isomorphisms
$$ \quo{A}{A \cap \mathbb{F}_2^n}, \quo{B}{B \cap \mathbb{F}_2^n} \cong \quo{\mathbb{F}_2^{n+1}}{\mathbb{F}_2^n} \cong \mathbb{F}_2 $$
This means that $A \cap \mathbb{F}_2^n$ (respectively, $B \cap \mathbb{F}_2^n$) is exactly the half of $A$ (respectively, $B$). Since $\{a_i\}_i \subseteq A \setminus \mathbb{F}_2^n$ and $\{b_j\}_j \subseteq B \setminus \mathbb{F}_2^n$, we can conclude that
$$ |\mathscr{A}||\mathscr{B}| = ab \le \frac{|A|}{2} \cdot \frac{|B|}{2} = 2^{\dim A + \dim B - 2} \le 2^{n-1} $$
This completes the proof. \qed

\section{HW 1.4}

Let's consider cyclic permutations consist of $\{1, 2, \cdots, n\}$, and we define a collection $P_i$ as a set of all cyclic permutations such that all elements of $A_i$ appears adjacently. By the similar method in the lecture, $|P_i| = |A_i|! \cdot (n-|A_i|)!$ holds for every $i$.

Let $\sigma$ be a cyclic permutation, and let $\mathscr{A}_\sigma$ be the set of all $A_i$'s containing $\sigma$. If we write $n(\sigma) := |\mathscr{A}_\sigma|$, then the next claim holds.

\begin{claim}\label{number of fixing sets}
    An inequality $n(\sigma) \le |A_i|$ holds for all $A_i \in \mathscr{A}_\sigma$.
\end{claim}

\begin{proof}[Proof of Claim~\ref{number of fixing sets}]
    Suppose that $A_i \in \mathscr{A}_\sigma$. For any $A_j \in \mathscr{A}_\sigma$, let's use $c_j$ (respectively, $c'_j$) to denote the most counter-clockwise (respectively, clockwise) number of $A_j$. Then, for any $A_j \in \mathscr{A}_\sigma$ different from $A_i$, the exactly one of $c_j$ and $c'_j$ appears in $A_i$ because the collection $\{A_i\}_i$ is an intersecting antichain such that $|A_i| \le \frac{n}{2}$ for all $i$. This is because if none of them appears in $A_i$, then either $A_i$ and $A_j$ are disjoint or $A_i \subseteq A_j$ holds, and if both of them appears in $A_i$, then either $\{A_i, A_j\}$ covers $\{1, 2, \cdots, n\}$ or $A_j \subseteq A_i$ holds.

    Let $a_k$ be the $k$th element of $A_i$ from counter-clockwise in the permutation $\sigma$. Let's make $2$ by $|A_i|$ matrix $M=(M_{i,j})_{(i,j)}$ such that
    $$ M_{1,j} =
    \begin{cases}
        1 & \text{if } a_j = c_i \text{ for some } i\\
        0 & \text{otherwise}
    \end{cases}
    $$
    and
    $$ M_{2,j} =
    \begin{cases}
      1 & \text{if } a_j = c'_i \text{ for some } i\\
      0 & \text{otherwise}
    \end{cases}
    $$
    Then, $M_{1,|A_i|}, M_{2,1} = 0$ by the antichain condition, and $M_{1,j}+M_{2,j-1} \le 1$ by the intersecting and length condition. This implies that $|\mathscr{A}_\sigma| = 1+M (1)_{|A_i| \times 1} \le |A_i|$, and this completes the proof.
\end{proof}

Therefore, we can conclude that
\begin{align*}
    (n-1)! &= \# \text{ of all cyclic permutations}\\
    &= \sum_{i=1}^{m} \sum_{\sigma \in P_i} \frac{1}{n(\sigma)}\\
    &\ge \sum_{i=1}^{m} \sum_{\sigma \in P_i} \frac{1}{|A_i|}\\
    &= \sum_{i=1}^{m} \frac{|P_i|}{|A_i|} = \sum_{i=1}^{m} \frac{(n-1)!}{\binom{n-1}{|A_i|-1}}
\end{align*}
or equivalently,
$$ \sum_{i=1}^{n} \frac{1}{\binom{n-1}{|A_i|-1}} \le 1 $$
holds. This completes the proof. \qed

\section{HW 1.5}

\begin{claim}\label{showing special cases are enough}
    To obtain the conclusion, it is enough to show only for the cases $\{1, \alpha_i\}_i$ is $\mathbb{Q}$-linearly independent.
\end{claim}

\begin{proof}[Proof of Claim~\ref{showing special cases are enough}]
    Note that there are compactly supported non-negative continuous functions $(f_i)_{i \in \mathbb{N}}$ such that for any half-plane $H_\alpha = H := L_{\alpha}^{-1}\left(\mathbb{R}_{\ge \frac{1}{2}}\right)$, where $L_{\alpha}:(x_i)_i \mapsto \sum_i \alpha_i x_i$ and $\alpha = (\alpha_i)_i \in \Delta_n := \left\{(a_i)_i \in [0,1]^n \,;\, \sum_i a_i = 1\right\}$, we have $\int_H f_i \, \dx \ge P(H)$ for every $i$, and $\int_H f_i \, \dx$ converges to $P(H)$ pointwisely, where $P$ is the given probability measure in the problem. This is because $P$ is a linear combination of Dirac delta measures, and a Dirac delta measure can be approximated by Borel measures of the form $f\, \dx$ where $f$ is a compactly supported non-negative continuous function. Also, the inequality of measures can be obtained by choosing the functions with appropriately large volume, and with sufficiently small support, and positioning it at the outside of the generalized cube intersecting with $\mathbb{R}_{\ge 0}^n$ but close to the each non-origin vertices.

    With this setting, for any $\alpha \in \Delta_n$ such that $\{1, \alpha_i\}_i$ are $\mathbb{Q}$-linearly independent, we have $\int_H f_i \, \dx \ge P(H) \ge p$. Also, the map $\Delta_n \to \mathbb{R}, (\alpha_i)_i \mapsto \int_H f_i \, \dx$ is continuous since if the support of $f_i$ is contained in a sufficiently large ball $B_R(0)$, then
    $$ \left| \int_{H_\alpha} f_i \, \dx - \int_{H_\beta} f_i \, \dx \right| \le m(B_R(0)) \cdot \sup f_i(x) \cdot \cos^{-1}(\alpha \cdot \beta) $$
    holds. However, it can be easily verified that the subset of $\Delta_n$ consist of all elements such that $\{1, \alpha_i\}_i$ is $\mathbb{Q}$-linearly independent is dense because the induction argument on $n$ with facts $\dim_\mathbb{Q} \mathbb{R} = \infty$ and the denseness of $\mathbb{Q}$ in a suitable sense works. Therefore, $\int_H f_i \, \dx \ge p$ for every $\alpha \in \Delta_n$. Since $\lim_i \int_H f_i \, \dx = P(H)$ for every $\alpha \in \Delta_n$, we can conclude that $P(H) \ge p$.
\end{proof}

Now, the remainder is to show for the cases $\{1, \alpha_i\}_i$ is $\mathbb{Q}$-linearly independent. Let $\mathscr{F}_k := \left\{X \in \mathscr{P}(1,2, \cdots, n) \,;\, |X|=k, \sum_{i \in X} \alpha_k \ge \frac{1}{2}\right\}$. Then, $\mathscr{F}_k$ is $k$-uniform and intersecting because if $X_1, X_2 \in \mathscr{F}_k$ are disjoint, then $\sum_{i \in X_1 \sqcup X_2} \alpha_i = 1$, but this contradicts to the fact that $\{1, \alpha_i\}_i$ is $\mathbb{Q}$-linearly independent. Similarly, for any $k$ element set $X \in \mathscr{P}(1,2, \cdots, n)$, the condition $X \in \mathscr{F}_k$ is equivalent to $X^c \notin \mathscr{F}_{n-k}$, in particular, $|\mathscr{F}_k|+|\mathscr{F}_{n-k}| = \binom{n}{k}$ holds. By the Erd\H{o}s--Ko--Rado theorem, $|\mathscr{F}_k| \le \binom{n-1}{k-1}$ provided that $k \le \frac{n}{2}$. Hence, we have

\begingroup
\allowdisplaybreaks
\begin{align*}
    P\left( \sum_i \alpha_i X_i \ge \frac{1}{2} \right) &= \sum_{k \in [0,n]} P\left( \sum_i X_i = k, \sum_i \alpha_i X_i \ge \frac{1}{2} \right)\\ 
    &= \sum_{k \in [0,k]} |\mathscr{F}_k| \cdot p^k (1-p)^{n-k}\\ 
    &= \sum_{k \in [0, \frac{n}{2}]} |\mathscr{F}_k| \cdot p^k (1-p)^{n-k} + \sum_{k \in ]\frac{n}{2}, n]} |\mathscr{F}_k| \cdot p^k (1-p)^{n-k}\\
    &= \sum_{k \in [0, \frac{n}{2}]} |\mathscr{F}_k| \cdot p^k (1-p)^{n-k} + {\color{red} \sum_{k \in [0\frac{n}{2}[} |\mathscr{F}_{n-k}| \cdot p^{n-k} (1-p)^k}\\
    &= \sum_{k \in [0, \frac{n}{2}]} |\mathscr{F}_k| \cdot p^k (1-p)^{n-k} + \sum_{k \in [0\frac{n}{2}[} {\color{red} \left( \binom{n}{k} - |\mathscr{F}_k| \right)} \cdot p^{n-k} (1-p)^k\\ 
    &= \left[\sum_{k \in [0,\frac{n}{2}[} |\mathscr{F}_k| \cdot \left( p^k (1-p)^{n-k} - p^{n-k} (1-p)^k \right) + \binom{n}{k} p^{n-k} (1-p)^k \right]\\
    & \hspace{21em} {\color{blue} + \binom{n}{\frac{n}{2}} p^{\frac{n}{1}} (1-p)^{\frac{n}{2}}}\\ 
    &\ge \left [\sum_{k \in [0,\frac{n}{2}[} {\color{red} \binom{n-1}{k-1}} \left( p^k (1-p)^{n-k} - p^{n-k} (1-p)^k \right) + \binom{n}{k} p^{n-k} (1-p)^k \right ]\\ 
    & \hspace{22.3em} {\color{blue} + \binom{n}{\frac{n}{2}} p^{\frac{n}{1}} (1-p)^{\frac{n}{2}}}\\ 
    &= \left[\sum_{k \in [0,\frac{n}{2}[} \binom{n-1}{k-1} p^k (1-p)^{n-k} + \binom{n-1}{k} p^{n-k} (1-p)^k \right] {\color{blue} + \binom{n}{\frac{n}{2}} p^{\frac{n}{1}} (1-p)^{\frac{n}{2}}}\\ 
    &= \sum_{k \in [0,\frac{n}{2}[} \binom{n-1}{k-1} p^k (1-p)^{n-k} + \sum_{k \in ]\frac{n}{2}, n]} \binom{n-1}{k-1} p^k (1-p)^{n-k} {\color{blue} + \binom{n}{\frac{n}{2}} p^{\frac{n}{1}} (1-p)^{\frac{n}{2}}}\\ 
    &\ge \sum_{k \in [0,n]} \binom{n-1}{k-1} p^k (1-p)^{n-k} = p
\end{align*}
\endgroup

holds where $k \in \mathbb{N}$ and the green terms have to be considered if and only if $n$ is an even number. This is because $\binom{n}{\frac{n}{2}} \ge \binom{n-1}{\frac{n}{2}-1}$ holds, and $p \in [\frac{1}{2}, 1]$ implies that $1-p \le p$, and so $p^k (1-p)^{n-k} - p^{n-k} (1-p)^k \le 0$ for all $k \in [0,\frac{n}{2}]$ holds. This completes the proof. \qed

\end{document}
