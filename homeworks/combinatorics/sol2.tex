\documentclass[a4paper,11pt]{article}
\usepackage{amsmath,amssymb,amsthm, tikz,titlesec,hyperref, mathrsfs, mathtools}
\usepackage[a4paper,margin=2cm]{geometry}
\usepackage{ulem}
\linespread{1.3}
\newtheorem{claim}{Claim}[section]

%%%%%%%%%%%%%%%%%%%%%%%% My Settings %%%%%%%%%%%%%%%%%%%%%%%%%

\hypersetup{
    colorlinks=true
}

\DeclareMathOperator*{\codim}{codim}
\DeclareMathOperator*{\im}{im}
\DeclareMathOperator*{\rank}{rank}
\DeclareMathOperator*{\Span}{Span}
\DeclareMathOperator*{\Sym}{Sym}

\newcommand{\dx}{\mathrm{d}x}

% Quotient
\newcommand*{\quo}[2]{ % \newfaktor{#1}{#2} -> #1/#2
  \raisebox{0.8\height}{\ensuremath{#1}} % Numerator
  \mkern-7mu\raisebox{-0.2\height}{\scalebox{2}{$\diagup$}}\mkern-7mu % Slash /
  \raisebox{-0.9\height}{\ensuremath{#2}} % Denominator
}

%%%%%%%%%%%%%%%%%%%%%%%%%%%%%%%%%%%%%%%%%%%%%%%%%%%%%%%%%%%%%%

\newcommand\name{Jeongwoo Park}   % Name of the student
\newcommand\university{KAIST} % Name of the university
\newcommand\department{Mathematical Sciences} % Name of the department
\newcommand\studentid{20190262} % Student ID

\title{KAIST\\2021 MAS575 Combinatorics\\
Homework\bigskip}
\author{\textbf{\Large \name} \\
University: \university\\
Department: \department\\
Student ID: \studentid}
\date{\today}

\begin{document}
\thispagestyle{empty}
\maketitle
\tableofcontents
\titleformat{\section}[frame]{\pagebreak}{\filright
\footnotesize  \enspace \textsf{KAIST --- MAS575 Combinatorics 2021 Spring}\enspace}{6pt}{\Large\bfseries\filcenter}

\section{HW 2.1}\label{2.1}

Let $\mathbf{a}_j \in \mathbb{F}_p^n$ (respectively, $\mathbf{b}_j$) be the $\mathbb{F}_p$-coefficient characteristic vector of $A_j$ (respectively, $B_j$) for all $1 \le j \le m$. If we define $n$-variable polynomials
$$ f_j(x_\bullet) := \sum_{l \in L} \left( x_\bullet \cdot \mathbf{b}_j - \bar l \right) $$
then we have an invertible lower triangular matrix $\left(f_{j} \left(\mathbf{a}_{j'}\right) \right)_{(j,j')} \in U_n(\mathbb{F}_p)$, because
$$ f_{j} \left(\mathbf{a}_{j'} \right) = 
\begin{cases}
  \sum_l \left(\overline{|A_j \cap B_j|} - \bar l \right) \in \mathbb{F}_p^\times & j = j'\\
  \sum_l \left(\overline{|A_{j'} \cap B_j|} - \bar l \right) = 0 & j > j'
\end{cases}
$$
holds. Let's consider induced polynomials $\bar f_j(x_\bullet) \in \mathbb{F}_p[x_i]_{1 \le i \le n}$ which obtained by changing $x_i^2$ to $x_i$ for each term of the polynomial $f_j(x_\bullet)$ for all $1 \le i \le n$, as in the lecture. Then, it is clear that $\left(f_{j}\left(\mathbf{a}_{j'}\right)\right) = \left(\bar f_{j}\left(\mathbf{a}_{j'}\right) \right)$ is invertible, in particular, the set of polynomials $\left\{ \bar f_j(x_\bullet) \right\}_{1 \le j \le m}$ is $\mathbb{F}_p$-linearly independent. Since every term of $\bar f_j(x_\bullet)$ is square-free, and $\deg \bar f_j(x_\bullet) \le |L| = s$ holds, so we can deduce that $f_j(x_\bullet) \in \langle x_\bullet^{\alpha_\bullet} \rangle_{|\alpha_\bullet|_\infty \le 1}$. By linear algebra, we can obtain that
$$ m = \left|\left\{ f_j(x_\bullet) \right\}_{1 \le j \le m}\right| \le \dim \langle x_\bullet^{\alpha_\bullet} \rangle_{|\alpha_\bullet|_\infty \le 1} = \sum_{i=0}^s \binom{n}{i} $$
This completes the proof. \qed

\section{HW 2.2}
We can label $\mathscr{F} = \left\{ A_j \right\}_{1 \le j \le m}$ to be $|A_1| \le \cdots \le |A_m|$ holds. Let $\mathbf{a}_j \in \mathbb{R}^n$ be the $\mathbb{R}$-coefficient characteristic vector of $A_j$, and we define $n$-variable polynomials
$$ f_j(x_\bullet) := \sum_{l \in L \,;\, l < |\mathbf{a}_j|_1} \left( x_\bullet \cdot \mathbf{a}_j - l \right) $$
for all $1 \le j \le m$. Then, we have an invertible upper triangular matrix $\left(f_j \left(\mathbf{a}_{j'} \right) \right)_{(j,j')} \in U_n(\mathbb{R})$, because
$$ f_j \left(\mathbf{a}_{j'} \right) = 
\begin{cases}
  \sum_{l < |\mathbf{a}_j|_1} \left(|\mathbf{a}_j|_1 - l \right) \in \mathbb{R}^\times & j=j'\\
  \sum_{l < |\mathbf{a}_j|_1} \left(|A_{j'} \cap A_j| - l \right) = 0 & j>j'
\end{cases}
$$
Also, we consider polynomials
$$ g_I(x_\bullet) := \left( \prod_{i=1}^r \left(|x_\bullet|_1-k_i \right) \right) \cdot \prod_{i \in I} x_i $$
for all $I \subseteq \left\{ 1,2, \cdots, n \right\}$ satisfying $|I| \le s-r$. Let's consider an indexing $\left\{ I \right\}_{|I|<s-r} = \left\{ I_\bullet \right\}$ such that $|I_1| \le |I_2| \le \cdots$. Then, the matrix $\left(g_{I_j} \left(\chi_{I_{j'}} \right) \right)_{(j,j')}$ is invertible upper triangular matrix where $\chi$ is the characteristic function, since
$$ g_{I_j} \left(\chi_{I_{j'}} \right)
\begin{cases}
  \in \mathbb{R}^\times & j=j' \text{ because } k_i > s-r \text{ for all } i.\\
  = 0 & j > j' \text{ because } \prod_{i \in I_j} \chi_{I_{j'}}(i) = 0.
\end{cases}
$$
Also, we have $g_{I_j} \left(\mathbf{a}_{j'} \right) = 0$, hence the matrix
$$
\begin{bmatrix}
  \left(f_j \left(\mathbf{a}_{j'} \right) \right)_{(j,j')} & \left(f_j \left(\chi_{I_{j'}} \right) \right)_{(j,j')}\\
  \left(g_{I_j} \left(\mathbf{a}_{j'} \right) \right)_{(j,j')} & \left(g_{I_j} \left(\chi_{I_{j'}} \right) \right)_{(j,j')}
\end{bmatrix} =
\begin{bmatrix}
  \left(f_j \left(\mathbf{a}_{j'} \right) \right)_{(j,j')} & \left(f_j \left(\chi_{I_{j'}} \right) \right)_{(j,j')}\\
  0 & \left(g_{I_j} \left(\chi_{I_{j'}} \right) \right)_{(j,j')}
\end{bmatrix}
$$
is an invertible upper triangular matrix. In particular, the set of polynomials $\left\{ f_j(x_\bullet), g_I(x_\bullet) \right\}_{\substack{1 \le i \le m \\ |I| \le s-r}}$ is $\mathbb{R}$-linearly independent.

Now, let $\bar f_j(x_\bullet)$ (respectively, $\bar g_I(x_\bullet$)) be the reduction of $f_j(x_\bullet)$ (respectively, $g_I(x_\bullet)$) as in the \textbf{Problem~\ref{2.1}}. Then, we have the same invertible upper triangular matrix
$$
\begin{bmatrix}
  \left(\bar f_j \left(\mathbf{a}_{j'} \right) \right)_{(j,j')} & \left(\bar f_j \left(\chi_{I_{j'}} \right) \right)_{(j,j')}\\
  \left(\bar g_{I_j} \left(\mathbf{a}_{j'} \right) \right)_{(j,j')} & \left(\bar g_{I_j} \left(\chi_{I_{j'}} \right) \right)_{(j,j')}
\end{bmatrix} =
\begin{bmatrix}
  \left(f_j \left(\mathbf{a}_{j'} \right) \right)_{(j,j')} & \left(f_j \left(\chi_{I_{j'}} \right) \right)_{(j,j')}\\
  \left(g_{I_j} \left(\mathbf{a}_{j'} \right) \right)_{(j,j')} & \left(g_{I_j} \left(\chi_{I_{j'}} \right) \right)_{(j,j')}
\end{bmatrix}
$$
Again, the set $\left\{ \bar f_j(x_\bullet), \bar g_I(x_\bullet) \right\}_{j, I}$ is linearly independent. Since each term of $\bar f_j(x_\bullet)$ and $\bar g_I(x_\bullet)$ is square-free, and by two inequalities $\deg \bar f_j(x_\bullet) \le s$, $\deg \bar g_I(x_\bullet) \le |I|+r \le s$, we can obtain that $\bar f_j, \bar g_I \in \langle x_\bullet^{\alpha_\bullet} \rangle_{|\alpha_\bullet|_\infty \le 1}$. Therefore, the inequality
$$ m + \sum_{i=0}^{s-r} \binom{n}{i} = \left| \left\{ \bar f_j(x_\bullet), \bar g_I(x_\bullet) \right\}_{j, I} \right| \le \dim \langle x_\bullet^{\alpha_\bullet} \rangle_{|\alpha_\bullet|_\infty \le 1} = \sum_{i=0}^{s} \binom{n}{i} $$
holds, or equivalently,
$$ m \le \sum_{i=s-r+1}^{s} \binom{n}{i} $$
holds. This completes the proof. \qed

\section{HW 2.3}
Let $A_i = \left\{ i \right\}$ and $B_i = \left\{ 1, 2, \cdots, i-1 \right\}$. Then, this satisfies the conditions (i) and (ii). Also, we have
$$ \sum_{i=1}^{m} \frac{1}{\binom{|A_i|+|B_i|}{|A_i}} = \sum_{i=1}^m \frac{1}{i} $$
If we pick sufficiently large $m$, then we can obtain the inequality
$$ \sum_{i=1}^m \frac{1}{\binom{|A_i|+|B_i|}{|A_i|}} \ge n $$
and this completes the proof. \qed

\section{HW 2.4}
Let's write $\mathscr{F} = \left\{ A_i \right\}_{1 \le i \le m}$, and let $\mathbf{a}_i \in \mathbb{F}_2^n$ be the $\mathbb{F}_2$-coefficient characteristic vector of $A_i$. Pick $i \in [1,m]$, then there are vectors $b_j \in \mathbb{F}_2^n$ where $1 \le j \le k$ such that
\begin{enumerate}
  \item Each coordinate-wise product $b_i b_{i'}$ is zero when $i \neq i'$, and the sum of all $b_i$'s is the vector consist of only $1$ as coordinates.
  \item For each $i'$, the equality $a_{i'} \cdot b_j = 1$ holds for all $j$ if and only if $i' = i$ holds.
\end{enumerate}
because of the coloring condition in the problem. By the symmetry, we can assume that the first coordinate of $b_1$ is $1$. In this situation, we define ($n-1$)-variable polynomial
$$ f_i(x_\bullet) = \prod_{j=2}^{k} (x_\bullet' \cdot b_j) $$
over $\mathbb{F}_2$, where $x_\bullet'$ is the vector obtained by deleting the first entry of the vector $x_\bullet$. If $i' \neq i$, then the number of a non-zero entry of the coordinate-wise product $a_{i'}b_j$ is either $0$ or $2$ for some $j \ge 2$, by the pigeonhole principle. In the other word, the equality $f_i(a_{i'}) = \delta_{i,i'}$ holds, where $\delta_{\bullet, \bullet}$ denotes a Kronecker-delta function, i.e., the set $\left\{ f_i(x_\bullet) \right\}_{1 \le i \le m}$ is $\mathbb{F}_2$-linearly independent. Note that $f_i(x_\bullet)$ is homogeneous of degree $k-1$ only consist of square-free terms, so $f_i(x_\bullet) \in \langle x_\bullet'^{\alpha_\bullet} \rangle_{|\alpha_\bullet|_1 = k-1}$ holds. Hence, the inequality
$$ m = \left| \left\{ f_i(x_\bullet) \right\}_{1 \le i \le m} \right| \le \dim \langle x_\bullet'^{\alpha_\bullet} \rangle_{|\alpha_\bullet|_1 = k-1} = \binom{n-1}{k-1} $$
holds, and this completes the proof. \qed

\section{HW 2.5}
\begin{claim}\label{expectation lemma}
  Let $\mathscr{F}_1, \mathscr{F}_2$ be two $k$-uniform families over $[n]$. Then, the equality $\mathbb{E}\left(|\mathscr{F}_1 \cap \sigma(\mathscr{F}_2)|\right) = \frac{|\mathscr{F}_1| \cdot |\mathscr{F}_2|}{\binom{n}{k}}$ holds, where the expectation is with respect to the uniform probability measure on the symmetric group $\Sym(n)$.
\end{claim}

\begin{proof}[Proof of Lemma~\ref{expectation lemma}]
  Easy calculation shows that
  \begin{align*}
    \mathbb{E}(|\mathscr{F}_1 \cap \sigma(\mathscr{F}_2)|) &= \sum_{A \in \mathscr{F}_1, B \in \mathscr{F}_2} \Pr(A = \sigma(B))\\
    &= \sum_{A \in \mathscr{F}_1, B \in \mathscr{F}_2} \frac{|\Sym(n)_A|}{n!}\\
    &= \sum_{A \in \mathscr{F}_1, B \in \mathscr{F}_2} \frac{k! (n-k)!}{n!}\\ 
    &= \frac{|\mathscr{F}_1| \cdot |\mathscr{F}_2|}{\binom{n}{k}}
  \end{align*}
  where $\Sym(n)_A$ denotes the stabilizer of $A$, and this proves the claim.
\end{proof}

We have to show that $\mathbb{E}(|\mathscr{F}_1 \cap \sigma(\mathscr{F}_2)|) \le 1$ if we set $\mathscr{F}_1$ and $\mathscr{F}_2$ as in the problem. It is enough to show that for each $\sigma \in \Sym(n)$, the inequality $|\mathscr{F}_1 \cap \sigma(\mathscr{F}_2)| \le 1$ holds. If not, let $A, B$ be two different sets in $\mathscr{F}_1 \cap \sigma(\mathscr{F}_2)$ for some $\sigma$. Then, both $|A \cap B| \in L_1$ and $|A \cap B| = |\sigma^{-1}(A) \cap \sigma^{-1}(B)| \in L_2$ holds since $\sigma^{-1}(A), \sigma^{-1}(B) \in \mathscr{F}_2$, but this contradicts to the assumption that the given two sets $L_1$ and $L_2$ are disjoint. Therefore, the inequality $|\mathscr{F}_1 \cap \sigma(\mathscr{F}_2)| \le 1$ holds for every $\sigma$, and this completes the proof. \qed

\end{document}